README for sam2p
by pts@fazekas.hu
at Sun Dec 30 19:30:17 CET 2001 -- Fri Mar 22 19:25:03 CET 2002
Sat Apr 27 00:39:12 CEST 2002
Wed Jul  3 01:20:40 CEST 2002
Wed Feb  5 19:46:51 CET 2003

This is the README file for sam2p, a raster to PostScript/PDF image
conversion program. This file contains a 5-minute turbo tutorial for new and
impatient users (search for the phrase `Turbo tutorial' in your text editor).
As of now, this README file is the only, and definitive documentation of sam2p.

sam2p is a UNIX command line utility written in ANSI C++ that converts many
raster (bitmap) image formats into Adobe PostScript or PDF files and several
other formats. The images are not vectorized. sam2p gives full control to
the user to specify standards-compliance, compression, and bit depths. In
some cases sam2p can compress an image 100 times smaller than the PostScript
output of many other common image converters. sam2p provides ZIP, RLE and
LZW (de)compression filters even on Level1 PostScript devices. 

A testimonail from Grant Ingram, UK: Anyway this is just a quick note to say
thanks for writing the sam2p utility which I am using to create EPS figures
of photographs for my thesis -- it works very well producing image sizes
that are some 3% of the ones produced by ImageMagick.

The author of sam2p recommends his program over other image image converts
because of the following reasons:

-- sam2p produces much smaller output
-- sam2p gives the user complete control over the data layout of the output
   image, including Compression, SampleFormat and TransferEncoding
-- sam2p is fast
-- sam2p doesn't depend on external libraries. (But it depends on external
   programs for _reading_ JPEG, TIFF and PNG files.)
-- sam2p supports the mainstream image formats of today without compromise.
   sam2p has many file format fine-tuning features that are missing from
   most other converter utilities. For example:
   TIFF ZIP compression, TIFF LZW compression, TIFF
   JPEG compression, transparent PNG files, BMP RLE-4 and RLE-8
   compression etc.
-- sam2p supports all levels (versions) of the PostScript language, and
   output images have the smallest file size allowed by the LanguageLevel.
-- PostScript ZIP, RLE and LZW compression is provided for _all_
   LanguageLevels (!), even for PSL1 appeared in 1980. You can print your
   ZIP-compressed images onto your ancient printer of the 1980s.
-- sam2p supports all versions of PDF, and
   output images have the smallest file size allowed by the version.
-- output images of sam2p are always compliant to the standard selected by
   the user
-- output images of sam2p are real-world compatible, i.e the author has
   tested them with many common image processing programs, for example:
   Ghostscript, pdfTeX, xpdf, Acrobat Reader, The GIMP, ImageMagick, xv,
   Acrobat Distiller, QuarkXPress, InDesign. The author has also tested
   PostScript files on HP and OkiData printers.
-- sam2p converts every pixel faithfully, preserving all the 24 RGB bits
   intact. There is no quality or information loss unless you ask for it.
-- sam2p uses only a minimal number of libraries. You don't have to install
   33Mb of ballast software to use sam2p. Image libraries (libtiff etc.) are
   _not_ used, the math library is not used, libstdc++ is not used, zlib is
   not used.

Long-term limitations:

-- only DeviceRGB color space, with the Indexed, Gray and RGB image types
-- Indexed images can have 0..256 colors
-- alpha channel and transparency supported only for Indexed images: only
   one color may be transparent
-- the entire input image is read into memory. During operation, both the
   input and the output images may be held in memory.
 
Status
~~~~~~
sam2p is currently beta software. It is available from:

	http://www.inf.bme.hu/~pts/sam2p/
	http://www.inf.bme.hu/~pts/sam2p-latest.tar.gz
	http://sam2p.sf.net/

The documentation is incomplete, but -- together with the examples -- it is
quite useful. Please have a look at the home page to find articles and more
documentation (the PDF docs are much more eye-pleasing than this README).
The source code contains valuable comments, but they may be
hard to find unless you're deeply developing sam2p.

The author is developing sam2p in his free time. (He is studying and
working in non-free time.)

The imaging model is complete, and image output routines are stable and
adequate. Reasonable defaults are provided for all command line options.
sam2p can find the best SampleFormat automatically, and there is
an educated (but not perfect) default guess for the Compression.

See subsection {OutputRule combinations} about all planned formats.

The most important short-term limitations:

-- The code hasn't been extensively tested. The author welcomes bug reports.
-- The code hasn't been profiled for speed bottlenecks. (Although it seems
   to be faster than ImageMagick convert(1) in most situations.)

Turbo tutorial
~~~~~~~~~~~~~~
Quick compilation instructions:

1. ./configure --enable-lzw --enable-gif
2. make
3. Copy the `sam2p' executable to your $PATH, or invoke it as `./sam2p'.

Quick try:

-- ./sam2p examples/pts2.pbm try.eps
-- ./sam2p examples/pts2.pbm try.pdf
-- ./sam2p examples/pts2.pbm try.ps
-- ./sam2p examples/pts2.pbm try.png
-- ./sam2p examples/pts2.pbm try.tiff
-- ./sam2p examples/pts2.pbm try.xpm
-- ./sam2p examples/pts2.pbm try.bmp
-- ./sam2p examples/pts2.pbm try.jpg

A really short User's guide
"""""""""""""""""""""""""""
To convert an image, call:

	./sam2p <INPUT.IMG> <OUTPUT.IMG>
	Example: ./sam2p examples/pts2.pbm try.eps

To print an image as a full PostScript page, call:

	./sam2p [MARGIN-SPECS] <INPUT.IMG> ps: - | lpr
	Example: ./sam2p -m:1cm examples/pts2.pbm ps: - | lpr

To convert an image to be included as EPS (Encapsulated PostScript) into
(La)TeX documents, call:

	./sam2p <INPUT.IMG> <OUTPUT.eps>
	Example ./sam2p examples/pts2.pbm test.eps

To convert an image to be included as PDF into pdf(La)TeX documents, call:

	./sam2p <INPUT.IMG> <OUTPUT.pdf>
	Example ./sam2p examples/pts2.pbm test.pdf

If you have a large image file (possibly originating from dumb software),
you can reduce the image size and keep the same filename. Call:

	./sam2p <INPUT-OUTPUT.IMG> --
	Example: ./sam2p test.tiff --

You may specify a compression method (or supply other command line options)
to make a file even smaller, call:

	./sam2p [OPTIONS] <INPUT.IMG> <OUTPUT.IMG>
	Example: ./sam2p -c:zip test.tiff test2.tiff

See the detailed documentation of available command-line options elsewhere
in this document. You may also read section {FAQ} for more information.

Too see a list about the supported input and output image file formats, call:

	./sam2p

Example output:

	This is sam2p v0.39.
	Available Loaders: JAI PNG JPEG TIFF PNM BMP GIF LBM XPM PCX TGA.
	Available Appliers: XWD Meta Empty BMP PNG TIFF6 TIFF6-JAI JPEG-JAI JPEG PNM GIF89a XPM PSL1C PSL23+PDF PDF-JAI PSL2-JAI l1fa85g P-TrOpBb.
	Usage: [...]

The list of ``Available Loaders'' lists the input image file formats. All
except for JAI are self-explanatory. JAI is JPEG-as-is, it means reading a
JPEG file and writing back the exactly same image into an other JPEG variant,
without quality loss.

From the list of ``Available Appliers'' one can derive the supported output
image file formats. XWD, BMP, PNG, TIFF6, JPEG, PNM, GIF89a and XPM are
self-explanatory. TIFF6-JAI, JPEG-JAI, PDF-JAI and PSL2-JAI are JPEG
variants into which JAI files (see above) can be saved. The names of the
remaining appliers are quite cryptic for the beginner user; most of those
appliers provide sam2p's excellent support for writing PS, EPS and PDF
files.

sam2p operation modes
~~~~~~~~~~~~~~~~~~~~~
sam2p is a command line utility (i.e, without a graphical user
interface), so it can be used by composing a command line with the
appropriate options and parameters, and launching it. See sections ``Turbo
tutorial'' and ``One-liner mode'' for more details.

sam2p is not interactive, it doesn't ask questions; thus it is completely
suitable for batch processing and automation. sam2p doesn't log errors, but
its STDERR can be redirected to a log file quite easily.

There are three modes sam2p can operate in:

-- one-liner mode: (since sam2p 0.37)
   the user has to type a long command line, specifying the input and output
   file name, output file format, compression options etc. Most of the
   functionality of sam2p is available in a quite intuitive way in one-liner
   mode. Users of the `convert' utility from ImageMagick and `tiff2ps' and
   `tiffcp' will find that one-liner mode of sam2p is very similar to all
   these. This mode is recommended for impatient users.

   Due to the nature of sam2p development, some new functionality of job mode
   might be missing from one-liner mode. Please report this as a bug.

-- job mode: the user has to write a ``job'' file (recommended extension:
   .job), which specifies all conversion parameters, including the input and
   output file name. The name of the job file must be passed to sam2p. This
   mode is recommended for expert users, who want to retain full control of
   all aspects of the final output. All functionality is available in job
   mode.

-- GUI mode: This is completely experimental, and will be very probably
   dropped in the near future. Try executing sam2p.tk (TCL/Tk is required).
   Please don't use GUI mode, use one-liner mode instead!
   No more documentation is provided for GUI mode.
   
   There will be a Micro$oft Windoze version of sam2p available in the near
   future, but very probably you won't get real GUI with radio boxes, lists
   and file selection dialogs. You'll have to start sam2p from the DOS
   prompt...

One-liner mode
~~~~~~~~~~~~~~
This section contains a reference-style summary of one-liner mode.
The author knows that this section is quite incomprehensible, and a bit old.
He is planning to completely rewrite it to be readable for the novice user. 

The order of the arguments and options is significant.

Input file extension is discarded, the file format is recognised by its
magic number.

Output file extension gives a hint for /FileFormat:

.ps .eps .epsi .epsf: EPS: PSL3|PSL2 PSLC PSL1
.pdf: PDF: PDFB1.2|PDFB1.0 PDF1.0 PDF1.2
.gif; GIF89a
.pnm; PNM (for use with transparency)
.pbm: PNM /SampleFormat/Gray1
.pgm: PNM /SampleFormat/Gray8
.ppm: PNM /SampleForamt/Rgb8
.pam: PAM
.pip: PIP
.empty: Empty
.meta: Meta
.jpeg .jpg: JPEG
.tiff .tif: TIFF
.png: PNG
.xpm: XPM
.bmp: BMP /Compression/RLE
.rle: BMP /Compression/RLE

Options (case insensitive):

-- -j -j:job : display in-memory .job file
-- -j:warn : be verbose and display warnings about impossible combinations in
   the .job file
-- -s:Indexed1:Indexed4:Indexed8: Try /SampleFormats in this order, and try
   all others after these. Can be
   specified separately (e.g `-s Indexed1 -s Indexed2:Indexed8')
-- -s:Indexed1:Indexed4:Indexed8:stop: Try only these /SampleFormats in
   this order. Can be
   specified separately (e.g `-s Indexed1:Indexed2 -s Indexed8:stop')
-- -s:Indexed1:Indexed4:Indexed8:stopq: Try only these /SampleFormats in
   this order, be quiet (no warnings on failures). Can be
   specified separately (e.g `-s Indexed1:Indexed2 -s Indexed8:stop')
-- -s:tr equivalent to `-s Transparent:Opaque:Mask:Transparent2:Transparent4:Transparent8'
-- -l:...: /LoadHints(...)
-- disabled: -a: /LoadHints(asis) extra /Compression/JAI; load JPEG files (and others as-is)

-- -1  -ps:1 PSL1:: [tiff2ps] hint /FileFormat/PSL1 among /PSL*
-- -1c -ps:1c -ps:c PSLC:: [pts] hint /FileFormat/PSLC among /PSL*
-- -2  -ps:2 PSL2: EPS2::: [tiff2ps,imagemagick] default hint /FileFormat/PSL2 among /PSL*
-- -3  -ps:3 PSL3:: [pts] hint /FileFormat/PSL3 among /PSL*
-- -pdf:b0 PDFB1.0:: [pts] hint /FileFormat/PDFB1.0 among /PDF*
-- -pdf:b2 PDFB1.2:: [pts] default hint /FileFormat/PDFB1.2 among /PDF*
-- -pdf:0 PDF1.0:: [pts] hint /FileFormat/PDF1.0 among /PDF*
-- -pdf:2 PDF1.2:: [pts] hint /FileFormat/PDF1.2 among /PDF*
-- EPS: EPSF:: [pts] hint /FileFormat/PSL2 or /FileFormat/PSL3 (for /Compression/ZIP)
-- PDF:: [pts] hint /FileFormat/PDFB1.0 or /FileFormat/PDFB1.2 (for /Compression/ZIP)
-- PS:: [pts] hint /Scale/RotateOK /FileFormat/PSL2 or /FileFormat/PSL3 (for /Compression/ZIP)
-- PS2:: [imagemagick] hint /Scale/RotateOK /FileFormat/PSL2. Deprecated,
   please use PS:.

-- -e:0 -e:none: /Scale/None
-- -e -e:1 -e:scale: /Scale/OK
-- -e:rot -e:rotate: /Scale/RotateOK

-- GIF: GIF89a:: [imagemagick,pts] /FileFormat/GIF89a
-- JPEG: JPG:: [imagemagick,pts] /FileFormat/JPEG
-- TIFF: TIF:: [imagemagick,pts] /FileFormat/TIFF
-- PNG:: [imagemagick] /FileFormat/PNG
-- XPM:: [imagemagick] /FileFormat/XPM
-- BMP:: [imagemagick] /FileFormat/BMP
-- Empty:: [pts] /FileFormat/Empty
-- Meta:: [pts] /FileFormat/Meta
-- PIP:: [pts] /FileFormat/PIP
-- PAM:: [pts] /FileFormat/PAM
-- PNM:: [imagemagick] /FileFormat/PNM (for use with transparency)
-- PBM:: [imagemagick] /FileFormat/PNM /SampleFormat/Gray1
-- PGM:: [imagemagick] /FileFormat/PNM /SampleFormat/Gray8
-- PPM:: [imagemagick] /FileFormat/PNM /SampleFormat/Rgb8

-- -t:bin: [pts] hint /TransferEncoding/Binary (default unless /PS*)
-- -t:hex: [pts] hint /TransferEncoding/Hex (default for /PSL1 /PSLC)
-- -t:a85: [pts] hint /TransferEncoding/A85 (default for /PSL2 /PSL3)
-- -t:ascii: [pts] hint /TransferEncoding/ASCII
-- -t:lsb1 -f:lsb2msb: [pts,tiffcp] hint /TransferEncoding/LSBfirst
-- -t:msb1 -f:msb2lsb: [pts,tiffcp] hint /TransferEncoding/MSBfirst

-- -c:none: [pts,tiffcp] non-default hint /Compression/None
-- -c:lzw: [pts,tiffcp] hint /Compression/LZW
-- -c:lzw:(1..99): [pts] hint /Compression/LZW /Predictor ...
-- -c:zip: [pts,tiffcp] hint /Compression/ZIP
-- -c:zip:(1..99): [pts] hint /Compression/ZIP /Predictor ...
-- -c:zip:(1..99):(-1..9): [pts] hint /Compression/ZIP /Predictor ... /Effort ...
-- -c:(rle|packbits): [pts,tiffcp] hint /Compression/RLE
-- -c:(rle|packbits):(0..): [pts] hint /Compression/RLE /RecordSize ...
-- -c:fax: [pts] hint /Compression/Fax
-- -c:fax:(-1..): [pts] hint /Compression/Fax /K ...
-- -c:dct: [pts] hint /Compression/DCT /DCT<<>>
-- -c:dct:...: [pts] hint /Compression/DCT /DCT<<...>>
-- -c:jpeg: [pts,tiffcp] hint /Compression/JAI, /Compression/IJG
-- -c:jpeg:(0..100): [pts] hint /Compression/JAI, /Compression/IJG /Quality ...
-- -c:ijg: [pts,tiffcp] hint /Compression/IJG
-- -c:ijg:(0..100): [pts] hint /Compression/IJG /Quality ...
-- -c:g4: [pts] equivalent to -c:fax:-1
-- -c:g3 -c:g3:1d: [pts] equivalent to -c:fax:0, -c:fax
-- -c:g3:2d: [pts] equivalent to -c:fax:-2
-- -c:jai: [pts] hint /Compression/JAI

-- -m:(dimen) -m:all:(dimen) -m:a:(dimen): set all margins (/TopMargin,
   /BottomMargin, /LeftMargin, /RightMargin) to `dimen'
-- -m:horiz:(dimen) -m:h:(dimen) -m:x:(dimen): set /LeftMargin and
   /Rightmargin to `dimen'
-- -m:vert:(dimen) -m:v:(dimen) -m:y:(dimen): set /TopMargin and
   /BottomMargin to `dimen'
-- -m:left:(dimen) -m:l:(dimen): set /LeftMargin to `dimen'
-- -m:right:(dimen) -m:r:(dimen): set /LeftMargin to `dimen'
-- -m:top:(dimen) -m:t:(dimen) -m:up:(dimen) -m:u:(dimen): set /TopMargin to
   `dimen'
-- -m:bottom:(dimen) -m:b:(dimen) -m:down:(dimen) -m:d:(dimen): set
   /BottomMargin to `dimen'

-- -- as last arg: OutputFile:=InputFile
-- -- earlier: treat other args as filenames

Default and fallback compression types for each file format:

-- PSL1 PSLC: /RLE
-- PSL2 PDFB1.0 PDF1.0: /JAI /RLE
-- PSL3 PDFB1.2 PDF1.2: /JAI /ZIP
-- GIF89a: /LZW
-- XPM PNM PAM PIP Empty Meta: )/None)
-- JPEG: /JAI /IJG
-- TIFF: /JAI /LZW? /RLE
-- PNG: /ZIP
-- BMP: /RLE

Overview of job mode
~~~~~~~~~~~~~~~~~~~~
In ``job mode'', sam2p doesn't accept any command line options: it must be
controlled from ``job'' files. sam2p expects a single command line
argument: the name of the Job file (file format described in section
{Jobs}). sam2p runs that single job, prints debug, info, notice, warning and
error messages (etc.), and creates a single output file: a PS or a PDF. For
multiple jobs and multiple output files, one has to run sam2p multiple
times.

The details about the output file format (including standards-compliance,
compression and transfer encoding) are specified in the Job file and other
files. Thus, in order to make use of the (basic and) advanced
features of sam2p in job mode, you have to:

1. Understand the basic concepts (i.e read through this manual, and have a
   look at the examples).
2. Prepare the input raster (bitmap) graphics file in one of the supported
   input formats (see section {Supported input formats}).
3. Decide the name of the output file.
4. Decide some or all details of the output format.
5. Create a Job file that describes those details.
6. Invoke the program `sam2p' with the name of the Job file as a single
   command-line argument (or `-' if the Job file is fed on STDIN).
7. Read warning and error messages (printed to STDOUT and STDERR), and retry
   if necessary.

Compilation and installation
~~~~~~~~~~~~~~~~~~~~~~~~~~~~
External software required for running sam2p:

-- a UNIX system with a fairly standard BSD or POSIX C library (C++
   libraries are not required), or a Win32 system with MSVCRT.DLL (install
   Wordpad to get MSVCRT.DLL)
-- optionally: the libjpeg `cjpeg' utility for /Compression/IJG
-- optionally: the libjpeg `djpeg' utility for reading JPEG files
-- optionally: tif22pnm (uses libtiff) from the author of sam2p
-- optionally: png22pnm (uses libpng) from the author of sam2p
-- optionally: the Ghostscript `gs' utility for /Compression/DCT
-- optionally: the Ghostscript `gs' utility for /Compression/Fax

These do not work yet:

-- optionally: the `lzw_codec' utility for alternative /Compression/LZW
-- optionally: the GNU `gzip' utility for alternative /Compression/ZIP
-- optionally: the Info-ZIP `zip' utility for alternative /Compression/ZIP

For Win32 compilation, see later.

Software required for UNIX compilation:

-- a UNIX system
-- a working, GNU-compatible C++ compiler (preferably GNU G++ >=2.91. Known
   working compilers: g++-2.91 g++-2.95 g++-3.0 g++-3.1 g++-3.2)
-- GNU Make (`make -v' should print `GNU Make')
-- Perl >=5.004 (no external Perl modules are required)
-- a Bourne-compatible shell (preferably GNU Bash >=2.0)
-- the following libraries are _not_ required: libjpeg, libtiff, libpng,
   libungif, PDFlib, zlib, libm, libstdc++
-- optionally: GNU autoconf >=2.53 (version number is important, see
   AC_C_CONST)

Compilation:

	# compile and install required programs
	autoconf  # optional, for experts only
	export CC=gcc-3.2 CXX=g++-3.2  # optional, for experts only
	./configure --enable-gif --enable-lzw
	make
	# the stand-alone utility `./sam2p' is now built
	make install  # optional, may not work

Testing:

	./sam2p
	./sam2p examples/ptsbanner_zip.job
	./sam2p examples/pts2.pbm try.eps
	gs test.ps
	# try other examples: examples/*.job

On Debian systems, you'll need GNU Make, Perl, GNU Bash and any of the
following packages for compilation:

	apt-get install libc6-dev gcc-2.95 g++-2.95
	apt-get install libc6-dev gcc-3.0 g++-3.0
	apt-get install libc6-dev gcc-3.1 g++-3.1
	apt-get install libc6-dev gcc-3.2 g++-3.2

Please also run any of the following before ./configure:

	export CC=gcc-2.95 CXX=g++-2.95
	export CC=gcc-3.0 CXX=g++-3.0
	export CC=gcc-3.1 CXX=g++-3.1
	export CC=gcc-3.2 CXX=g++-3.2

Optionally, you may install any of

	apt-get install gccchecker
	apt-get install autoconf

sam2p has been tested with a wide variety of GNU C++ compilers, including
g++-2.91, g++-2.95, g++-3.0, g++-3.1, g++-3.2, i386-uclibc-g++-2.95,
checkerg++-2.95. The program must be compilable _without_ _warnings_ with
any of g++-2.91, g++-2.95, g++-3.0, g++-3.1, g++-3.2. If there is a
compilation error, send a brief e-mail to the author immediately!

Portability
~~~~~~~~~~~
sam2p is quite portable on UNIX systems. It runs on:

	Debian GNU/Linux Slink  2.2.13 glibc-2.0.7 (development platform)
	Debian GNU/Linux Potato 2.2.18 glibc-2.1.3
	Debian GNU/Linux Sid    2.4.17 glibc-2.2.5
	Digital OSF1 V4.0 1229 alpha
	SunOS 5.7 Generic_106541-17 sun4u sparc SUNW,Ultra-2 gcc-2.95.2 
	SunOS 5.8 Generic_108528-12 sun4u sparc gcc-3.0.4

Also it runs on Win32 in command line and GUI mode.

It should work on any Linux or BSD system without modification. Porting to
other Unices should be quite easy. The author welcomes portability patches.

Porting to non-UNIX systems may be hard. Reasons:

-- Those systems might not have GNU Make, Perl or a Bourne-compatible shell
   installed. So the Makefile supplied won't work, and many human work would
   be necessary.
-- sam2p uses the popen(3) library call to communicate with external
   processes. This call might not be available on non-UNIX systems.
-- sam2p expects that the $PATH contains the external binaries. Some systems
   tend to have empty or misconfigured $PATH. On some systems, `gs' is
   called `gswin32c.exe' etc.

sam2p 0.38 has been compiled and run successfully on:

-- Linux 2.2.8 Debian Slink, g++-2.91
-- Linux 2.4.18-ac3 Debian SID, g++-2.95, g++-3.0, g++-3.1, g++-3.2
   Executable size: 318kB.
-- Linux 2.4.2 Debian Potato, gcc version 2.95.2 20000220 (Debian GNU/Linux)
   No warnings.
   Compilation took 0:47, executable size: 330kB.
-- Linux 2.2.16-3 Red Hat Linux release 6.2 (Zoot), gcc version egcs-2.91.66 19990314/Linux (egcs-1.1.2 release)
   No warnings.
   Compilation took 0:44, executable size: 324kB.
-- OSF1 V4.0 564 alpha, gcc version 2.7.2.2
   (tons of: warning: cast discards `const' from pointer target type,
    tons of: warning: the meaning of `\x' varies with -traditional
    tons of: warning: cast increases required alignment of target type)
   Compilation took 5 minutes, executable size: 550kB.
-- SunOS 5.7 Generic_106541-19 sun4u sparc SUNW,Ultra-2, gcc version 3.1
   (some: warning: cast from `char*' to `int*' increases required alignment of target type)
   Compilation took 2:50, executable size: 437kB.
-- SunOS 5.8 Generic_108528-15 sun4u sparc, gcc version 3.1.1
   (some: warning: cast from `char*' to `int*' increases required alignment of target type)
   Compilation took 1:26, executable size: 437kB.

sam2p 0.42 has been compiled and run successfully on:

-- Windows 98, Visual C++ 6.0
-- Windows 98, MSYS, MingGW, G++ 3.2

Win32 compilation instructions for command-line mode
~~~~~~~~~~~~~~~~~~~~~~~~~~~~~~~~~~~~~~~~~~~~~~~~~~~~
To compile sam2p.exe, the Win32 equivalent of the UNIX utility sam2p, you
have to install these build dependencies first:

-- MinGW and MSYS, available from http://www.mingw.org

-- Perl 5.004 or newer (only perl.exe and perl5*.dll is required), available
   from http://www.perl.com. Note that this will be a long download and a
   bloated install, but after that, just copy perl.exe and the single
   perl5*.dll to your C:\WINDOWS directory, and uninstall the rest.

To build sam2p:

1. Install all the build dependencies.

2. Open the MSYS terminal window from the start menu.

3. Run `explorer .' to figure out what is the current working directory.
   Let's call this directory the MSYS home.

4. Download the sam2p sources into the MSYS home:

	http://www.inf.bme.hu/~pts/sam2p-latest.tar.gz

5. Unpack the sources. Run:

	tar xzvf sam2p-latest.tar.gz
	tar xvf sam2p-latest.tar.gz # if the previous one doesn't work

6. Run `cd sam2p-*.*' to enter the sam2p source directory. It should contain
   a newer version of this README and the file sam2p_main.cpp.

7. Run `perl -edie' to check whether Perl is correctly installed. It should
   print a line beginning with `Died '. If no such line appears (or you get
   a `command not found' error message), go and install Perl first. Run
   `echo $PATH' to find out where MSYS is searching for perl.exe. Copy
   perl.exe to one of those directories.

8. Run

	./configure --enable-gif --enable-lzw
	make

9. The file sam2p.exe is now created in the current directory. Use it. You
   may copy it to another directory right now:

	cp sam2p.exe 'C:\Program Files'

10. You should invoke sam2p.exe from the command line (COMMAND.COM or
    CMD.EXE) with the _appropriate_
    arguments, described elsewhere in this document. Don't put it into the
    Start menu, it won't work (a window will flash in, showing an error
    message that you haven't supplied the right arguments).

11. The file bts2.tth is also created. It is an important file, because it
    is required for the GUI compilation.

12. Don't forget to install tif22pnm.exe to load TIFF files, djpeg.exe to
    load JPEG files, cjpeg.exe to save JPEG files, and png22pnm.exe to load
    PNG files. The installation instructions for these programs are not
    given here.

Win32 compilation instructions for GUI mode
~~~~~~~~~~~~~~~~~~~~~~~~~~~~~~~~~~~~~~~~~~~
vcsam2p.exe is a preliminary, alpha-stage attempt to provide a Win32 GUI for
sam2p.exe. Currently it can load and display images, but not cannot save
them. vcsam2p.exe is not ready for production use.

You'll need Visual Studio 6.0 installed.

1. Download the sam2p sources:

	http://www.inf.bme.hu/~pts/sam2p-latest.tar.gz

2. Download untarka.exe to be able to unpack the sources:

	http://www.inf.bme.hu/~pts/untarka.exe

3. Unpack the sources. Run:

	untarka.exe sam2p-latest.tar.gz

   A directory sam2p-*.* will be created, containing a newer version of this
   README and the file config-vc6.h

4. You'll need bts2.tth. You can get an old, possibly outdated and buggy
   version directly:

	http://www.inf.bme.hu/~pts/bts2.tth

   Or, you may compile sam2p under Linux (or Win32 command-line), and copy
   the generated bts2.tth from there.

   Copy bts2.tth to the same directory as config-vc6.h

5. Start the Visual C++ 6.0 environment.

6. File / Open Workspace / File type: Projects
                           Filename: vcsam2p.dsp
   Build / Set Active Configuration: vcsam2p - Win32 Release
   Build / Build vcsam2p.exe
   Build / Execute vcsam2p.exe

7. Don't forget to install tif22pnm.exe to load TIFF files, djpeg.exe to
   load JPEG files, cjpeg.exe to save JPEG files, and png22pnm.exe to load
   PNG files. The installation instructions for these programs are not
   given here.

Please report and fix bugs in vcsam2p.exe

Copyright
~~~~~~~~~
sam2p is written and owned by Szab� P�ter <pts@fazekas.hu>. sam2p contains
code from various people.

sam2p may be used, modified and redistributed only under the terms of the
GNU General Public License, found in the file COPYING in the distribution,
or at
                                                                                
	http://www.fsf.org/licenses/gpl.html
                                           
Supported input formats
~~~~~~~~~~~~~~~~~~~~~~~
-- PNM, PBM, PGM, PPM (preferred formats for non-transparent images)
-- PNM+PGM, PNM+PBM. The input is a concatenation of a PNM and a P[GB]M
   file with the same dimensions. The second P[GB]M contains the alpha
   channel.
-- XPM (preferred formats for indexed images with transparency)
-- BMP
-- GIF, with transparency
-- LBM (IFF ILBM), with transparency
-- TGA (Targa)
-- baseline JPEG JFIF (limited by /Compression/JAI)
-- PCX
-- JPEG, is supported with libjpeg/djpeg
-- TIFF, is supported with the author's tif22pnm; with transparency
-- PNG, is supported with libpng/pngtopnm (Debian package
   graphics/pnmtopng); with transparency

Note that only the major features of these file formats are supported. sam2p
is able to load most of these files, but not all of them.

Important, but unsupported input formats:

-- PS/EPS/PDF, may be supported with gs in the future (Note: EPS and PDF
   _output_ is already supported.)
-- TIFF, may be supported with libtiff/tiff2ps(!)/tifftopnm in the future
-- PNG, may be supported without libpng/pngtopnm (Debian package
   graphics/pnmtopng) in the future
-- XBM
-- XWD
-- Utah RLE

Input image model
~~~~~~~~~~~~~~~~~
A (sampled, raster, bitmap) image is a rectangular array of pixels (dots)
plus some metadata. Each pixel is represented by an unsigned integer which
is BPC (BitsPerComponent) this CPP (ComponentsPerPixel) wide. The image
coordinate system (X,Y): upper left corner is (0,0), upper right corner is
(Width-1,0), lower right corner is (Width-1,Height-1). (Note that this is
the natural, traditional top->down, left->right system, and it is different
from PostScript and PDF!).

Some pixels of the image may be without color: they're transparent. A
transparent pixel is not painted, so whatever was left under it on the
paper, remains visible. (On the other hand, a colored pixel overrides the
pixel below unconditionally. E.g a white pixel overrides a black pixel, a
half-gray pixel, and also another white pixel; but a transparent pixel
leaves the original one visible.). Notions referring to transparent pixels
are: transparency, opacity, transparent, opaque, alpha channel, matte
channel.

Images are read from image files on disk. The file format is autodetected
(see section {Supported input formats}), and it can also be specified in the
Job file (NOT implemented yet). Not all file formats are able to specify all
pixel data and metadata, so additional hints (such as the transparent color
or the name of the image author) can be specified in Job files.

Sample formats
~~~~~~~~~~~~~~
The image pixels could be packed to bytes according to several sample
formats. Each output file (both EPS and PDF) has its own SampleFormat
(notation: capitals).

A color is either transparent or it is an opaque RGB triplet (8*3 bits).

The number of colors is the number of colors actually _used_. So unused
palette entries, and e.g unused #555555 in gray-4 are not counted.

If PSLC is required, but the printer is only PSL1, then the color image will
be printed grayscale.

When _choosing_ the output format, sam2p doesn't degrade image quality. For
example, if an image has only two colors: #000001 and #ffffff, sam2p won't
allow the gray-1 sample format, but with #000000 and #ffffff, it will. The
user is expected to have an image editor in which she can adjust image
colors precisely (such as in the Dialogs/(Indexed palette) dialog of The
GIMP).

Supported Sample Formats:

Name:
	Fast compatibility
	Slow compatibility
	Criteria for the image
	-- Comment(...)

transparent: (specialisation of mask)
	all
	-
	the whole image is transparent
	-- implemented with empty image body
opaque: (specialisation of mask and indexed-1)
	all
	-
	the whole image contains the same, opaque color
	-- implemented with `setrgbcolor', `fill'
mask: (specialisation of transparent-2)
	all
	-
	a transparent and a non-transparent color (any may be missing)
	-- display a Warning if the whole image is transparent or opaque,
	   because transparent or opaque would be a better choice
	-- implemented with a single call to `imagemask'
indexed-1:
	all
	-
	exactly 2 non-transparent colors or 1 non-transparent color
	-- display a Warning if only 1 non-transparent color, because
	   opaque would be a better choice
	-- display a Notice if colors are in black (#000000) and white
	   (#ffffff), beacuse gray-1 would be a better choice
	-- implemented with a `setrgbcolor', `fill', and a single call to
	   `imagemask'
indexed-2:
	PSL2, PDF1.0??
	PSLC
	3 or 4 non-transparent colors or 1..2 non-transparent colors
	-- display a Warning if only 1..2 non-transparent colors, because
	   opaque or indexed-1 would be a better choice
	-- display a Notice if colors are in (#000000, #555555, #aaaaaa,
	   #ffffff), beacuse gray-2 would be a better choice
	-- implemented with the /Indexed color space or colorimage + 
	   manual palette lookup
	-- users with a PSL1 printer without PSLC should use transparent-*
indexed-4:
	PSL2, PDF1.0??
	PSLC	
	5..16 non-transparent colors or 1..4 non-transparent colors
	-- display a Warning if only 1..4 non-transparent colors, because
	   opaque, indexed-1 or indexed-2 would be a better choice
	-- display a Warning if all components are #00 or #ff,
	   because rgb-1 would be a better choice (3 bits over 4 bits)
	-- display a Notice if colors are in (#000000, #111111, ...,
	   #ffffff), beacuse gray-4 would be a better choice
	-- implemented with the /Indexed color space or colorimage + 
	   manual palette lookup
	-- users with a PSL1 printer without PSLC should use transparent-*
indexed-8:
	PSL2, PDF1.0??
	PSLC
	17..256 non-transparent colors or 1..16 non-transparent colors
	-- display a Warning if only 1..16 non-transparent colors, because
	   opaque, indexed-1, indexed-2, indexed-4 would be a better
	   choice
	-- display a Warning if all components are #00, #55, #aa or #ff,
	   because rgb-2 would be a better choice (6 bits over 8 bits)
	-- display a Notice if all colors are gray, beacuse gray-8 would be
	   a better choice
	-- implemented with the /Indexed color space or colorimage + 
	   manual palette lookup
	-- users with a PSL1 printer without PSLC should use transparent-*
transparent-2:
	all
	-
	0..1 transparent and 1..3 non-transparent colors
	-- display a Notice that color separation was done (which can
	   decrease speed and compression)
	-- display a Warning if no transparent color, because `indexed-2'
	   would be a better choice
	-- display a Warning if only 1 non-transparent color, because `mask'
	   would be a better choice
	-- implemented with multiple calls to `setrgbcolor', `imagemask'
transparent-4:
	all
	-
	a transparent and 1..15 non-transparent colors
	-- display a Notice that color separation was done (which can
	   seriously decrease speed and compression)
	-- display a Warning if only 1..3 non-transparent colors, because
	   `mask' or `transparent-2' would be a better choice
	-- implemented with multiple calls to `setrgbcolor', `imagemask'
transparent-8:
	all
	-
	a transparent and 1..255 non-transparent colors
	-- display a Warning that color separation was done (which can
	   seriously decrease speed and compression)
	-- display a Warning if only 1..15 non-transparent colors, because
	   `mask', `transparent-2' or `transparent-4' would be a better
	   choice
	-- implemented with multiple calls to `setrgbcolor', `imagemask'
gray-1:
	all
	-
	colors are in black (#000000) and white (#ffffff)
	-- display a Warning if only 1 color, because opaque would be a
	   better choice
	-- implemented with the multiple-argument `image'
gray-2:
	all
	-
	colors are in (#000000, #555555, #aaaaaa, #ffffff)
	-- display a Warning if only 1..2 colors, because opaque,
	   indexed-1, or gray-1 would be a better choice
	-- implemented with the multiple-argument `image'
gray-4:
	all
	-
	colors are in (#000000, #111111, ..., #ffffff)
	-- display a Warning if only 1..4 colors, because opaque,
	   indexed-1, gray-1, indexed-2 or gray-2 would be a better choice
	-- implemented with the multiple-argument `image'
gray-8:
	all
	-
	colors must be gray
	-- display a Warning if only 1..16 colors, because opaque,
	   indexed-1, gray-1, indexed-2, gray-2, indexed-4 or gray-4
	   would be a better choice
	-- implemented with the multiple-argument `image'
rgb-1:
	PSLC, PDF1.0
	-
	color components must be #00 or #ff (8 colors max)
	-- display a Warning if all colors are gray
	-- display a Warning if only 1..4 colors, because opaque,
	   indexed-1, indexed-2 (or gray-*) would be a better choice
	-- implemented with `colorimage'
rgb-2:
	PSLC, PDF1.0
	-
	color components must be #00, #55, #aa or #ff (64 colors max)
	-- display a Warning if all colors are gray
	-- display a Warning if only 1..16 colors, because opaque,
	   indexed-1, indexed-2 or indexed-4 (or gray-*) would be a better choice
	   choice (this includes the case when color components are in
	   #00, #ff)
	-- implemented with `colorimage'
rgb-4:
	PSLC, PDF1.0
	-
	color components must be #00, #11, ... #ff (4096 colors max)
	-- display a Warning if all colors are gray
	-- display a Warning if only 1..256 colors, because opaque,
	   indexed-1, indexed-2, indexed-4 or indexed-8 (or gray-*) would be a better
	   choice (this includes the case when color components are in
	   #00, #55, #aa, #ff)
	-- implemented with `colorimage'
rgb-8:
	PSLC, PDF1.0
	-
	no transparency
	-- display a Warning if all colors are gray
	-- display a Warning if only 1..256 colors, because opaque,
	   indexed-1, indexed-2, indexed-4 or indexed-8 (or gray-*) would be a better
	   choice
	-- display a Warning if all color components are in
	   #00, #11, ... #ff, because rgb-4 would be a better choice
	-- implemented with `colorimage'

The following directed (acyclic) graph represents that some formats should
be tried earlier than others to avoid most Warning and Notice messages. The
graph was created according to the descriptions above.

	EarlierFormat LaterFormat

	transparent mask                                      
	opaque mask
	opaque indexed-1
	indexed-1 indexed-2
	indexed-2 indexed-4
	indexed-4 indexed-8
	gray-1 gray-2
	gray-2 gray-4 
	gray-4 gray-8
	rgb-1 rgb-2
	rgb-2 rgb-4
	rgb-4 rgb-8
	gray-1 indexed-1
	gray-2 indexed-2
	gray-4 indexed-4
	gray-8 indexed-8  
	rgb-1 indexed-4   
	rgb-2 indexed-8   
	mask transparent-2
	transparent-2 transparent-4
	transparent-4 transparent-8
	opaque gray-1   
	indexed-1 gray-2
	indexed-2 gray-4
	indexed-4 gray-8
	opaque rgb-1   
	gray-1 rgb-1
	gray-2 rgb-2    
	gray-4 rgb-4    
	gray-8 rgb-8    
	indexed-2 rgb-1
	indexed-4 rgb-2 
	indexed-8 rgb-4
	indexed-8 rgb-8

Every directed acyclic graph (DAG) has a topological ordering on its nodes.
Such an ordering can be computed by the UNIX (Version 7 AT&T UNIX) utility
tsort(1). Its output on the author's machine:

	opaque
	transparent
	gray-1
	indexed-1
	mask
	transparent-2
	gray-2
	indexed-2
	transparent-4
	rgb-1
	gray-4
	indexed-4
	transparent-8
	rgb-2
	gray-8
	indexed-8
	rgb-4
	rgb-8

This ordering should be taken into account when someone develops her
Rule Profile. Rules having SampleFormats listed earlier should be earlier in
the Rule Profile to avoid Warning and Notice messages.

The availability (and also Warnings and Notices) of a Sample Format for a
particular image can be easily decided after answering the following
characteristic questions:

-- Is transparency _used_?
-- How many _used_ non-transparent colors are there? (257 if >=257)
-- Is there a non-gray color?
-- How many bits are required (maximum) for each component?

Output rules
~~~~~~~~~~~~
Every detail of the output file format is precisely determined by the Output
Rule. The Output Rule may be specified in the Job file, or is
automatically chosen from several pre-defined output rules in the Output
Profile (see section {Output profiles} elsewhere in this document).

Output rule entries:

-- FileFormat: enum (see section {Standards} for detailed information), no
   default
   /PSL1 -- PostScript Level1
   /PSLC -- PostScript Level1 with the CMYK and `colorimage' extension
   /PSL2 -- PostScript Level2 (default)
   /PSL3 -- PostScript Level3
   /PDFB1.0 -- PDF version 1.0, BI inline image, see 4.8.6 in PDFRef.pdf
   /PDFB1.2 -- PDF version 1.2, BI inline image, see 4.8.6 in PDFRef.pdf
   /PDF1.0 -- PDF version 1.0, XObject image, see 4.8.4 in PDFRef.pdf
   /PDF1.2 -- PDF version 1.2, XObject image, see 4.8.4 in PDFRef.pdf
   /GIF89a
   /Empty
   /Meta
   /PNM
   /PAM
   /PIP
   /TIFF
   /JPEG
   /PNG
   /XPM

-- SampleFormat: enum, no default, see section {Sample formats}
   /Opaque
   /Transparent
   /Gray1
   /Indexed1
   /Mask
   /Transparent2
   /Gray2
   /Indexed2
   /Transparent4
   /Rgb1
   /Gray4
   /Indexed4
   /Transparent8
   /Rgb2
   /Gray8
   /Indexed8
   /Rgb4
   /Rgb8
   /Asis -- accept contents of the JAI file
   /Bbox -- no image, only bounding box information
-- WarningOK: boolean; this Output Rule is enabled iff WarningOK is true or
   SampleFormat causes no warnings, default: true
-- TransferEncoding: enum, no default
   /Binary -- Binary (RawBits, see pbm(5), pgm(5), ppm(5)) (Binary integers
   are stored in any byte order allowed by /FileFormat)
   /ASCII -- ASCII (text, chars: 9,10,13,32..126), used with transparent and opaque
   /Hex /AHx -- Hex ((PSL1), PDF1.0, PSL2 ASCIIHexEncode filter)
   /A85 -- A85 (PSL2 PDF1.0, ASCII85Encode filter)
   /MSBfirst -- Binary data with integers stored in MSB first byte order.
     If 0x41424344 is represented as "ABCD", the byte order is called: big
     endian, MSB, MSB first (preferred), most significant byte first, most
     significant bit first, MSB-to-LSB, network byte order, m68k byte order.
     QuarkXPress 3 can read only TIFF files with MSB-to-LSB byte order.
   /LSBfirst -- Binary data with integers stored in LSB first byte order.
     If 0x41424344 is represented as "DCBA", the byte order is called:
     little endian, LSB, LSB first (preferred), least significant byte
     first, least significant bit first, LSB-to-MSB, VAX byte order, PC
     (i386) byte order.
-- Compression: enum
   /None -- None (default)
   /LZW -- LZW (PSL2 PDF1.0 LZWEncode filter EarlyChange=true, UnitLength=8
     LowBitFirst=false)
   /ZIP /Flate /Fl -- ZIP (PSL3 PDF1.2 FlateEncode filter without options)
   /RLE /RunLength /RunLengthEncoded /RL /PackBits -- RLE (PSL2 PDF1.0
     RunLengthEncode filter, similar to TIFF PackBits)
   /Fax /CCITTFax /CCF -- Fax (PSL2 PDF1.0 CCITTFaxEncode filter,
     Uncompressed=true!, K=-1,0,1, EndOfLine=false, EncodedByteAlign=false,
     Columns=..., Rows=0, EndOfBlock=true, BlackIs1=false,
     DamagedRowsBeforeError=0)
   /DCT -- DCT (PSL2 PDF1.0 DCTEncode, options in JPEG
     stream)
   /IJG /JPEG /JPG /JFIF -- IJG (PSL2 PDF1.0 DCTEncode, options in JPEG
     stream; the IJG libjpeg library is used for compression, respecting the
     quality value 0..100). This requires /SampleFormat/Rgb8 or
     /SampleFormat/Gray8. This doesn't work with /SampleFormat/Asis.
   /JAI -- JAI (PSL2 PDF1.0 DCTEncode, options in JPEG stream; JPEG-as-is: the
     input file must be a JPEG file -- its contents are transferred
     unmodified into the /DCTDecode JPEG stream). This requires
     /SampleFormat/Asis, and doesn't work with any other /SampleFormats
-- Predictor: enum (see later), numbering same as PSL1 filter.
   1 -- no predictor. (default) Must be this unless Compression is /LZW or
        /Flate
   2 -- TIFF predictor 2 (horizontal differencing)
   10 -- PNG predictor, None function
   11 -- PNG predictor, Sub function
   12 -- PNG predictor, Up function
   13 -- PNG predictor, Average function
   14 -- PNG predictor, Paeth function
   15 -- PNG predcitor, individually chosen for each line (signed minimum)
   45 -- PNG predcitor, individually chosen for each line (unsigned minimum)
-- Transparent: color. Default: null. Specify a color forced to be
   transparent. Old transparency, if exists, is blacked!
-- Hints: dict
   see below

The Hints member of the Output Rule contains a dict with the following
elements:

-- TopMargin: dimen; desired vertical gap between the top line of the page
   and the top line of the raster. Default: 0. Ignored unless for PS and PDF
   output. See docs about `dimen' elsewhere in this document.
-- BottomMargin: dimen; desired vertical gap between the bottom line of the
   raster and the bottom line of the page. Default: 0. Ignored unless for PS
   and PDF output. See docs about `dimen' elsewhere in this document.
-- LeftMargin: dimen; desired horizontal gap between the left line of the page
   and the left line of the raster. Default: 0. Ignored unless for PS and PDF
   output. See docs about `dimen' elsewhere in this document.
-- RightMargin: dimen; desired horizontal gap between the right line of the
   raster and the right line of the page. Default: 0. Ignored unless for PS
   and PDF output. See docs about `dimen' elsewhere in this document.
-- Scale: enum
   /None -- don't scale (zoom, magnify) the image (default)
   /OK -- scale PS image to fit page (x factor == y factor)
   /RotateOK -- scale and/or rotate PS image to fit page (x factor == y factor)
-- EncoderBPL: int >=1 (bits per scanline, <= rlen)
-- EncoderCoumns: int >=1 (pixels per scanline)
-- EncoderRows: int >=1
-- EncoderColors: int >=1
-- PredictorColumns: uint; also used if compression is /Fax (reasonable
   default)
-- PredictorColors: 1..3; number of color _components_ (reasonable default)
-- PredictorBPC: 1, 2, 4, 8 (reasonable default),
   /BitsPerComponent entry in PS and PDF
-- Effort: -1..9, must be -1 unless Compression is /ZIP (-1 means
   5, default)
-- RecordSize: uint, default: 0. Compression must be /RLE
-- K: int, default: 0 (-2..infty). Compression must be /Fax. -1 means G4
   1d encoding, 0 meangs G3 1D encoding, -2 means G3 2D encoding with
   arbitrary height, positive value means G3 2D encoding with that height.
-- Quality: 0..100, used by IJG libjpeg when compression is
   /IJG. default: 75
-- ColorTransform: 0..2. For IJG, this _must_ be 0 for Gray and 1 for RGB,
   so its value is ignored. For DCT, its value is respected: use 0 or 1
   only.
   See DCTEncode in subsubsection 3.13.3 in PLRM.pdf, and for a better
   documentation: see the sources and docs of libjpeg.
-- TransferCPL: number of data characters per line. Must be positive
   when TransferEncoding is /Hex or /A85, and must be zero otherwise.
   Default: 78
-- DCT: dict, default: <<>>. Additional parameters for the /DCTEncode filter
-- Comment: string, default: empty
-- Title: string, default: empty
-- Subject: string, default: empty
-- Author: string, default: empty
-- Creator: string, default: empty
-- Producer: string, default: empty
-- Created: string, default: now
-- Produced: string: default: now

Metric units
""""""""""""
Certain parameters have type `dimen'. This is a metric dimension, measured
in any of the following real-word distance metric units:

--  1 bp = 1 bp (big point)
--  1 in = 72 bp (inch)
--  1 pt = 72/72.27 bp (point)
--  1 pc = 12*72/72.27 bp (pica)
--  1 dd = 1238/1157*72/72.27 bp (didot point) [about 1.06601110141206 bp]
--  1 cc = 12*1238/1157*72/72.27 bp (cicero)
--  1 sp = 72/72.27/65536 bp (scaled point)
--  1 cm = 72/2.54 bp (centimeter)
--  1 mm = 7.2/2.54 bp (millimeter)

Each image pixel is assumed to be 1 bp wide and 1 bp tall. A dimen is an
integer or real number, followed by optional whitespace and an optional unit
(any of `bp', `in', `pt', `pc', `dd', `cc', `sp', `cm', `mm'). The default
unit is `bp', i.e a bare number is a dimen measured in `bp'. The following
dimens are all one inch long: `72', `72bp', `72 bp', `1in', `1 in',
`2.54cm', `25.4mm', `72.27pt', `6pc', `4736286.72sp'.

Note: MiniPS and TeX use the same units.

OutputRule combinations
~~~~~~~~~~~~~~~~~~~~~~~
In the final version of sam2p, the following combinations will be supported:

LZW >=2 Binary|Hex|A85 >=PSL2|>=PDF1.0 Mask|Gray*|RGB*|Indexed*
LZW >=2 Binary|Hex|A85 >=PSL2|>=PDF1.0 Transparent+
ZIP >=2 Binary|Hex|A85 >=PSL3|>=PDF1.2 Mask|Gray*|RGB*|Indexed*
ZIP >=2 Binary|Hex|A85 >=PSL3|>=PDF1.2 Transparent+
None|ZIP|LZW|RLE|Fax|DCT|IJG 1 Binary|Hex|A85 >=PSL2|>=PDF1.0 Mask|Gray*|RGB*|Indexed*
None|ZIP|LZW|RLE|Fax|DCT|IJG 1 Binary|Hex|A85 >=PSL2|>=PDF1.0 Transparent+
None 1 ASCII >=PSL1|>=PDF1.0 Opaque
None 1 ASCII >=PSL1|>=PDF1.0 Transparent
ZIP  1 Binary|Hex|A85 >=PSL1 Gray*|RGB*|Indexed*
ZIP  1 Binary|Hex|A85 >=PSL1 Mask|Gray1|Indexed1
ZIP  1 Binary|Hex|A85 >=PSL1 Transparent+
None 1 Binary|Hex|A85 >=PSL1 Gray*|RGB*|Indexed*
None 1 Binary|Hex|A85 >=PSL1 Mask|Gray1|Indexed1
None 1 Binary|Hex|A85 >=PSL1 Transparent+
RLE  1 Binary|Hex|A85 >=PSL1 Gray*|RGB*|Indexed*
RLE  1 Binary|Hex|A85 >=PSL1 Mask|Gray1|Indexed1
RLE  1 Binary|Hex|A85 >=PSL1 Transparent+
JAI  1 Binary|Hex|A85 >=PSL2|>=PDF1.0 Asis

TTM files
~~~~~~~~~
TTM stands for Template Toy Macro.

A TTM file is a dirty hack for generating templates with auto-calculated
lengths and offsets. Currently they are used for generating PDF output files
(/FileFormat/PDFB10 etc.). The syntax
of a TTM file is MiniPS (i.e a minimalistic PostScript, similar to .job
files). The TTM file must contain a single MiniPS array.

The elements of the array are called chunks. Each chunk causes some bytes
to be appended to the output file. Data is appended in the order the
chunks are listed in the TTM file, but the data calculation order may be
different. This way it is possible to write the length of a chunk not
written yet. The very first chunk has number zero.

The chunk types:

-- string: backtick-sequences will be substituted (e.g ``w' to the width of
   the image, in pixels) by writeTemplate(). The result is appended to the
   output file.
-- positive integer: The offset (zero-based byte-position of the very first
   character of chunk 0) of the specified chunk will be appended to
   the output file. Only chunks already appearead may be specified this
   way. If the specified chunk is an array, then printf("%10u") will be
   called to print the number (this is useful for making PDF xref tables),
   otherwise printf("%u") will be called.
-- negative integer: The length (measured in bytes, after substitutions)
   of the specified chunk will be appended to the output file, using
   printf("%u"). Only chunks already calculated may be specified this way.
-- zero: error
-- array: the array is interpreted as a standalone TTM subfile, and the rules
   are applied recursively. This subfile contains sub-chunks, and the
   subchunks may be arrays themselves.
-- other MiniPS types: error

The chunks are calculated in the following order: first the array chunks are
calculated (recursively) in order of appearance, followed by the non-array
chunks in order of appearance.

A TTM file can have up to 64 top-level chunks.

Example:

	[ 1              %0
	  [ (pts) ]      %1.0
	  -1             %2
	]

The output file will be: `3pts0000000001' since chunk 1 has length 3 and
offset 1.

Example job file
~~~~~~~~~~~~~~~~
	<<%sam2p job
	% This is file (named test0.job).
	/InputFile  (test0.pbm)
	/OutputFile (test0.pdf)
	/Profile [
	  % This in-line profile is preferred over the defaults
	  << /FileFormat/PDF10 /SampleFormat/Gray1 /TransferEncoding/Binary
	     /Compression/Fax /Hints<</K 99>> >>
	  (pdf10.jib) run  % elements found in external file
	]
	>>

See the directory examples/*.job in the sam2p sources.

FAQ
~~~
Q1. Should I care about /LoadHints (,asis,) when loading JPEG files?

A1. No, sam2p guesses it by magic (in both job mode and one-liner mode).
    However, you may want to set it manually in job mode:

	/LoadHints () % use djpeg
	/LoadHints (,asis,) % don't use djpeg
	% nothing: automatic guess, based on /Compression/JAI

Q2. How do I convert a JPEG file to PostScript Level2 EPS?

A2. In one-liner mode, just run:

	./sam2p <INPUT.jpg> <OUTPUT.eps>
	Example: ./sam2p try.jpg try.eps

    In one-liner mode, if you have both the djpeg and cjpeg utilities
    (budled with libjpeg from IJG (Independent JPEG Group)), _and_ you want
    to adjust quality vs size of the output, just run:

	./sam2p -c:jpeg:<QUALITY> <INPUT.jpg> <OUTPUT.eps>
	Example: ./sam2p -c:jpeg:60 try.jpg try.eps

    In job mode, just run sam2p with the following .job file:

	<<%sam2p-job; 
	% conversion is possible without external utilities cjpeg and djpeg
	% No quality loss, just verbatim adata copying.
	/InputFile  (INPUT.jpg)
	/OutputFile (OUTPUT.eps)
	/Profile [
	  << /FileFormat/PSL2 /SampleFormat/Asis /TransferEncoding/A85
	     /Compression/JAI >>
	] >>

    Alternatively, to adjust quality vs size, use the following .job file:

	<<%sam2p-job;
	% external utilities cjpeg and djpeg are required
	% This uses a JPEG decompression (djpeg), plus lossy JPEG compression
	% (cjpeg), so there might be quality loss!
	/InputFile  (INPUT.jpg)
	/OutputFile (OUTPUT.eps)
	/Profile [
	  << /FileFormat/PSL2 /SampleFormat/Rgb8 /TransferEncoding/A85
	     /Compression/IJG /Hints <<
	       /Quality 40 % 0..100 (should be at least around 30)
	  >> >>
	] >>

Q3. How do I convert a GIF file to PostScript Level2 EPS?

A3. Check that sam2p has been compiled with GIF support: run sam2p, and
    examine its console output. It should contain a line:

	Available Loaders: ... GIF ...

    If GIF doesn't appear in the line, please recompile sam2p with:

	make clean
	./configure --enable-gif --enable-lzw
	make
	cp sam2p /usr/local/bin

    After that, run sam2p again, and check for the line above again.

    In one-liner mode, just run:

	./sam2p <INPUT.gif> <OUTPUT.eps>
	Example: ./sam2p try.gif try.eps

    In job mode, if the GIF file doesn't have transparent pixels, run sam2p
    with the following .job file:

	<<%sam2p-job; 
	/InputFile  (INPUT.gif)
	/OutputFile (OUTPUT.eps)
	/Profile [
	  << /FileFormat/PSL2 /SampleFormat/Indexed8 /TransferEncoding/A85
	     /Compression/None >>
	] >>

    If the GIF file has transparent pixels, run sam2p with the following .job
    file:

	<<%sam2p-job; 
	/InputFile  (INPUT.gif)
	/OutputFile (OUTPUT.eps)
	/Profile [
	  << /FileFormat/PSL2 /SampleFormat/Transparent8 /TransferEncoding/A85
	     /Compression/None >>
	] >>

Q4. How do I covert a JPEG file to a TIFF/JPEG output file?

A4. A TIFF/JPEG file is a TIFF file (_not_ a JPEG file!), in which the image
    data is compressed with JPEG (DCTEncode compression). The Compression
    TIFF tag value is 7. (There is also Compression==6, which corresponds to
    the old, obsolete JPEG format defined in the old TIFF6.0 spec.)

    In one-liner mode, autodetection is magical. Just run:

	./sam2p <INPUT.jpg> <OUTPUT.tiff>
	Example: ./sam2p try.jpg try.tiff

    In job mode, run sam2p with the following .job file:

	<<%sam2p-job;
	/InputFile  (INPUT.jpg)
	/OutputFile (OUTPUT.tiff)
	%/LoadHints (asis) % default for /Compression/JAI
	/Profile [
	  << /FileFormat/TIFF /SampleFormat/Asis /TransferEncoding/Binary
	     /Compression/JAI >>
	] >>

    See also {FAQ question Q5} for compatibility notes.

Q5. The TIFF/JPEG file generated by sam2p is invalid! I cannot read it with
    any programs.

A5. No, it isn't invalid, but most of the programs (including those found in
    libtiff) cannot deal with TIFF files with JPEG compression.

    Compatibility notes:

    -- tif22pnm 0.03 (from the author of sam2p) can read TIFF/JPEG files
       perfectly. That's because it calls the TIFFRGBAImageGet() function
       of libtiff, which works.

    -- sam2p 0.37 can read TIFF/JPEG files, beacuse it calls tif22pnm to do
       the job. Sam2p can write TIFF/JPEG files as well.

    -- GIMP 1.0.2: error message: `Unknown photometric number 6'. GIMP TIFF
       import filter cannot deal with the YCbCr color space (which is the
       most common and de facto standard color space in non-grayscale JPEG
       files). It works, however, with grayscale JPEGs.

    -- tifftopnm from libtiff-tools 3.4beta037-5.1: `unknown photometric:
       6'. Ditto. (Unfortunately tifftopnm doesn't call TIFFRGBAImageGet(),
       but it tries to re-implement an obsolete version of the function.)

    -- `tiffcp -c jpeg' from libtiff-tools 3.4beta037-5.1 creates a
       perfectly legal TIFF/JPEG file.

    -- tiffcp from libtiff-tools 3.4beta037-5.1 cannot load a file created
       by itself (`tiffcp -c jpeg')! There is no problem with grayscale
       images, but color images have one component removed.

    -- xv 3.10a: Ditto.

    -- display from ImageMagick 4.04: strange error message about libraries:
       `JPEGLib: Wrong JPEG library version: library is 61, caller expects 62.'

    Simple conclusion:

    -- Use sam2p or `tiffcp -c jpeg' to create a TIFF/JPEG. (Be aware that
       `tiffcp -c jpeg' cannot read a TIFF/JPEG: it can only create one.)
    -- Use tif22pnm to load or decode a TIFF/JPEG.
    -- In your own C programs, call the TIFFRGBAImageGet() function to read
       TIFF image data.
    -- Don't use anything else if you want to avoid compatibility problems.

Q6. Does sam2p support transparency and alpha channels?

A6. sam2p supports only bilevel transparency (i.e a pixel is either fully
    opaque or fully transparent), and only with indexed images. Transparency
    is supported when loading indexed PNG, TIFF, PNM, GIF, LBM and XPM files.
    A PNM file with transparency is a regular PBM/PGM/PPM file with a
    PBM image appended to it as the alpha channel (black pixel is
    transparent).

    For transparent output, the user has to specify /Transparent, /Mask,
    /Transparent2, /Transparent4 or /Transparent8 as /SampleFormat. This
    works with:

    -- /FileFormat/PSL1+    /SampleFormat/Transparent
    -- /FileFormat/PDF1.0+  /SampleFormat/Transparent
    -- /FileFormat/PDFB1.0+ /SampleFormat/Transparent
    -- /FileFormat/PSL1+    /SampleFormat/Mask
    -- /FileFormat/PDF1.0+  /SampleFormat/Mask
    -- /FileFormat/PDFB1.0+ /SampleFormat/Mask
    -- /FileFormat/GIF89a   /SampleFormat/Mask
    -- /FileFormat/PNM      /SampleFormat/Mask
    -- /FileFormat/TIFF     /SampleFormat/Mask
    -- /FileFormat/PNG      /SampleFormat/Mask
    -- /FileFormat/XPM      /SampleFormat/Mask
    -- /FileFormat/PSL1+    /SampleFromat/Transparent+
    -- /FileFormat/GIF89a   /SampleFormat/Transparent+
    -- /FileFormat/PNM      /SampleFormat/Transparent+
    -- /FileFormat/TIFF     /SampleFormat/Transparent+
    -- /FileFormat/PNG      /SampleFormat/Transparent+
    -- /FileFormat/XPM      /SampleFormat/Transparent+

Q7. How large is a pixel of PostScript and PDF files generated by sam2p in
    real-world metric units (inches or centimeters)?

A7. 72 big points == 1 inch == 2.54 centimeters

    1 pixel == 1 big point

Q8. I have an image with transparent pixels. What happens if I convert it to
    /Rgb* or /Gray*?

A8. Either of the following will happen:

    -- You get an error message, sam2p refuses to ignore transparency.
       Please use /SampleFormat/Transparent+, or call an image manipulation
       program to remove transparency from the image before feeding it to
       sam2p.
    -- Transparency information will be lost, and the color of formerly
       transparent pixels will be undefined. This would be a bug in sam2p,
       you should report it.

    However, if you loaded a GIF file, and
    transformed it to /Gray8 or /Rgb8, the original palette entry (RGB
    triplet) is faithfully preserved.

Q9. How do I generate a PostScript page ready for immediate printing with
    margins and the image properly scaled to fit the page?
    How do I create a PostScript file that will automatically scale the
    image to the maximum when printed?


A9. To print an image as a full PostScript page, call:

	./sam2p [MARGIN-SPECS] <INPUT.IMG> ps: - | lpr
	Example: ./sam2p -m:1cm examples/pts2.pbm ps: - | lpr

    To create a PostScript file for printing, call:

	./sam2p [MARGIN-SPECS] <INPUT.IMG> [ps:] <OUTPUT.ps>
	Example: ./sam2p -m:1cm examples/pts2.pbm try.ps

    The `-m' option above is sets all four margins to `1 cm'. You can 
    set the margins individually:

	Example: ./sam2p -m:left:7mm -m:right:1cm -m:top:0.5in \
          -m:bottom:18bp examples/pts2.pbm try.ps

    As you can see in this example, you may specify dimensions in various
    metric units, see subsection {Metric units}.

    You are strongly encouraged to print raster images with sam2p. Be aware
    that The GIMP 1.2 printing plugin has several weird contrast setting
    problems (even for /Gray1 images); white pixels will be gray etc. Other
    utilities may add unnecessary text banners or scale the image
    inappropriately.

    In one-liner mode, sam2p guesses from the file extension and the selector
    (`ps:') whether the desired output file format is PostScript (fit single
    page) or Encapsulated PostScript (leave size as-is, suitable for
    inclusion into TeX documents).

    In job mode, without /Scale/OK and /Scale/RotateOK in /Hints,
    sam2p outputs EPS (Encapsulated PostScript) with /FileFormat/PSL*. EPS
    files should be included as figures into other documents (such as TeX
    and InDesign), not printed alone. If you just want to print a sampled
    image alone, please use your favourite graphics manipulation program
    instead of sam2p.

    In job mode, create a .job file for the EPS file, and add /Hints. For
    example:

     <<%sam2p-job;
     /InputFile  (test.in)
     /OutputFile (test.ps)
     /Profile [
       << /FileFormat/PSL2 /SampleFormat/Rgb8 /TransferEncoding/A85
          /Compression/None /Predictor 1
          /Hints << /Scale/OK % or /Scale/RotateOK
                    /LeftMargin 12 % measured in inch/72
                    /Rightargin 12 % measured in inch/72
                    /TopMargin 12 % measured in inch/72
                    /BottomMargin 12 % measured in inch/72
                 >>
       >>
      ]
     >>

Q10. Do the EPS files created by sam2p conform to some specifications?

     The EPS output of sam2p conforms the following Adobe specifications:

	5001.DSC_Spec.pdf
	5002.EPSF_Spec.pdf

     DSC and ADSC are: Adobe Document Structuring Conventions. They are
     comments with lines beginning with `%!' and `%%' in PS and EPS files.

     An excerpt:

	The following example illustrates the proper use of DSC comments in a
	typical page description that an application might produce when including an
	EPS file. For an EPS file that is represented as

	%!PS-Adobe-3.0 EPSF-3.0
	%%BoundingBox: 4 4 608 407
	%%Title: (ARTWORK.EPS)
	%%CreationDate: (10/17/89) (5:04 PM)
	%%EndComments
	...PostScript code for illustration..
	showpage
	%%EOF

     DSC comments discussion:

	%!PS-Adobe-3.0 EPSF-3.0 (mandatory)
	%%BoundingBox: ... ... ... ... (mandatory)

  	%%Extensions: CMYK (optional, for /PSLC)
	%%LanguageLevel: 2 (optional, for /PSL2)
	%%LanguageLevel: 3 (optional, for /PSL3)
	%%Creation (strongly recommended)
	%%Title (strongly recommended)
	%%CreationDate (strongly recommended)
	%%Trailer (optional)
	%%EOF (optional)
	%%DocumentData: Clean7Bit (optional)
	%%DocumentData: Binary (optional)

Q11. I get the error message `sam2p: Error: applyProfile: invalid
     combination, no applicable OutputRule'. Help!

A11. This error message means you have requested an invalid combination of
     FileFormat, SampleFormat, Compression etc. parameters. If you use
     one-liner mode, and you're sure that you've specified your will
     correctly in the command line, please report this error message as a
     sam2p bug (also specify -j in the command line). If you use job mode,
     please read on.

     Example 1:
     /Compression/Fax is not allowed in /PSL1. Example 2: /Compression/IJG
     requires /SampleFormat/Gray8 or /SampleFormat/Rgb8. Please have a look
     at the messages `sam2p: Warning: check_rule: ...' to get more specific
     information. After that, correct your request. Solution 1: specify
     /FileFormat/PSL2 /Compression/Fax. Solution 2: specify /Compression/IJG
     /SampleFormat/Rgb8.

     An other cause for this message is that your request cannot be applied
     to the image you've specified. In this case, there is no relevant
     `sam2p: Warning: check_rule: ...' message.
     Example 1: you've requested
     /SampleFormat/Indexed4, but the input image has more than 16 colors.
     Example 2: you've requested /SampleFormat/Indexed4, but the input image
     has transparency. Solution 1: specify /SampleFormat/Rgb8. Solution 2:
     specify /SampleFormat/Transparent8.

     It is possible, but very unlikely that this error message is caused by
     a bug in sam2p.

Q12. Can I use /Compression/Fax when bits-per-pixel > 1 ?

A12. With /FileFormat/PS* and /FileFormat/PDF*, you can (but you shouldn't,
     because of the possibly poor compression ratio). With /FileFormat/TIFF,
     you're not allowed to, because the TIFF specification forbids it.
     Example one-liners:

	sam2p -s:Indexed8 -c:fax test.gif test.pdf   # OK
	sam2p -s:Indexed8 -c:fax test.gif test.eps   # OK
	sam2p -s:Indexed8 -c:fax test.gif test.tiff  # forbidden

Q13. Bad luck?

A13. Not for me.

Q14. Can I use negative margins (i.e /TopMargin -20) to crop the output
     image?

A14. No. Margins are ignored by sam2p unless /FileFormat is /PSL* or /PDF*.
     Even with these formats, the image is only moved, not cropped. Please
     use an image manipulation program (e.g The GIMP) to crop your images
     before feeding them to sam2p.

Q15. When I try to print the PostScript output of sam2p, the edge of the
     image is missing (white).

A15. Many printers cannot print to the edge of the paper (so that region is
     left white). Please increase the margins to a safe value, for example:

	./sam2p -m:7mm test.ppm test.ps
	lpr test.ps

     See also {FAQ question Q9} for more information about margins.

Q16. How do I report a bug in sam2p?

A16. Please send an e-mail to the author (pts@fazekas.hu, see more in
     section {Copyright}) describing the problem. Don't forget to:

     -- download the latest version of sam2p, and try it with the same image
     -- describe what sam2p does (incorrectly)
     -- describe what sam2p should do if there was no bug
     -- run sam2p without arguments, and attach its output (STDOUT) to the
        bug report
     -- attach the exact command line with which you call sam2p to the bug
        report
     -- if you spot the bug in one-liner mode, specify the `-j' option in
        the command line, and attach the messages printed by sam2p (both
        STDOUT and STDERR) to the bug report
     -- if you spot the bug in job mode, attach the .job file you are using
        to the bug report
     -- attach the input image file to the bug report. Try to attach a file
        as small as possible.
     -- if sam2p runs successfully (i.e it prints `Success.'), and it
        creates an output image, but you think that the output image is
        incorrect, attach the output image to your bug report
     -- if you have a similar input image, for which sam2p works fine,
        attach it to the bug report

Q17. How long does the LZW patent held by Unisys last?

A17. mcb@cloanto.com (author of http://lzw.info) wrote:

     Thank you for your interest and mail. I must stress that the "exact"
     answers you may be looking for may come only from lawyers and courts,
     and I am none of these. If you consider the IBM, the BT and Unisys US
     patents, then the last of the three would be the Unisys one, expiring,
     as the article I think mentions, on June 19, 2003, 24:00. There cannot
     be other (new) patents on LZW, as far as I know. Please let me know if
     you find different information.

Q18. I want to create an RGB PostScript image, but sam2p creates a Gray one,
     or it gives me an error message.
     For example: `./sam2p -1 -s:rgb1 examples/ptsbanner.gif test.eps'.

A18. /PSL1 doesn't support RGB images. There are two solutions:

     -- Use /PSLC or /PSLC instead. Examples:

	./sam2p -1c -s:rgb1 examples/ptsbanner.gif test.eps  # /PSLC
	./sam2p -2  -s:rgb1 examples/ptsbanner.gif test.eps  # /PSL2
	./sam2p     -s:rgb1 examples/ptsbanner.gif test.eps  # /PSL2 or /PSL3

     -- Use /Mask or /Transparent+. Note that you'll very probably get poor
        compression ratio.

	./sam2p -1 -s:tr:stop examples/ptsbanner.gif test.eps  # /PSL1

     You can get more (and more useful) error messages from sam2p if you
     specify to `-j:warn' options. You may also try specifying
     `-s:rgb1:stop' instead of `-s:rgb1' to force sam2p try /SampleFormat/Rgb1
     only.

Q19. sam2p doesn't allow me to use /Compression /Fax. For example:
     `./sam2p -c fax examples/ptsbanner.gif test.eps'. The same command
     works fine without `-c fax'.

A19. /Compression/Fax is intended to be used with images with 1 bit per
     pixel. However, in PostScript and PDF, you can use it for any image
     data, but compression ratio will be very poor for other than
     /Gray1, /Indexed1 or /Mask, of course. You can force sam2p to use /Fax
     by specifying the desired SampleFormat in option `-s'. Examples:

	sam2p -s:Indexed8 -c:fax test.gif test.pdf   # OK
	sam2p -s:Indexed8 -c:fax test.gif test.eps   # OK
	sam2p -s:Indexed8 -c:fax test.gif test.tiff  # forbidden by TIFF std
     
     See {FAQ question Q12} for more information.

Q20. Can sam2p convert images with transparency to PDF?

A20. Only if the image has at most 1 non-transprent color
     (/SampleFormat/Mask). See {FAQ question Q6} for details.
     
     Although PDF-1.3 supports transparency masks for arbitrary PDF images,
     sam2p 0.39 doesn't. That's because the author of sam2p hasn't
     implemented it yet.

Q21. I get the error message `sam2p: Warning: buildProfile: ignoring, no
     handlers for OutputRule'. Help!

A21. This means that sam2p doesn't know how to do the conversion you've
     requested (and it even doesn't know whether the request is erroneous or
     not). This might be because your request is bad (it is impossible to
     be fulfilled), or your request is good, but sam2p doesn't know how to
     deal with it. If you think that the latter is the case, please report
     this message as a bug.

     See {FAQ question Q11} for more information.

Q22. How do I compile with G++ 3.2?

A22. See the answer in section {Compilation and installation}. Don't forget

	export CC=gcc-3.2 CXX=g++-3.2

Q23. How do I do a `make dist' without running configure again?

A23. Just issue

	make MAKE_DIST=1 dist

Q24. I cannot open a JPEG file in the Win32 version.

A24. Make sure you have djpeg.exe on your PATH. Simply copy it to your
     C:\WINDOWS directory.

Q25. I cannot open a TIFF file in the Win32 version.

A25. Make sure you have tif22pnm.exe on your PATH. Simply copy it to your
     C:\WINDOWS directory.

Q26. I cannot open a PNG file in the Win32 version.

A26. Make sure you have png22pnm.exe on your PATH. Simply copy it to your
     C:\WINDOWS directory.

Q27. What is tif22pnm?

A27. tif22pnm is a TIFF -> PNM converter written by the author of sam2p. It
     can load more TIFF files correctly than tifftopnm, ImageMagick convert,
     xv and The GIMP. The TIFF loader code is based on GIMP 1.3, but has
     many bugfixes and improvements. sam2p uses tif22pnm to load TIFF files.
     You can download tif22pnm from

	http://www.inf.bme.hu/~pts/tif22pnm-latest.tar.gz

Q28. What is png22pnm?

A28. png22pnm is a PNG -> PNM converter compiled by the author of sam2p. It
     is based on the excellent pngtopnm utility, but doesn't depend on the
     NetPBM library (only libpng). sam2p png22pnm (or, as a fallback:
     pngtopm) to load PNG files. png22pnm is part of the tif22pnm package,
     so you can download it from

	http://www.inf.bme.hu/~pts/tif22pnm-latest.tar.gz

Q29. Can sam2p convert a transparent GIF to PDF?

A29. The PDF-1.3 file format supports transparent images, but sam2p doesn't.
     However, if the image contains at most two colors (including the
     transparent pixel), sam2p can create a working PDF-1.2 file; use
     Ghostscript to view it, because Acrobat Reader 5.0 is buggy. However,
     sam2p supports generating transparent EPS, GIF, PNG, PNM, XPM and TIFF
     files up to 256 colors.

Q30. How do I build my own sam2p debian package?

A30. Please download the newest sources from

	http://www.inf.bme.hu/~pts/sam2p-latest.tar.gz

     As root, run

	apt-get update
	apt-get install debmake fakeroot dpkg
	apt-get install make g++ gcc perl sed

     As normal user, run (in the directory containing sam2p_main.cpp):

	debian/rules clean
	rm -f build*
	debian/rules build
	fakeroot debian/rules binary
	ls -l ../sam2p_*.deb

     As root, substitute X and Y, and run:

	dpkg -i sam2p_X_Y.deb

     Please also install the tif22pnm and png22pnm packages from the author
     of sam2p (and the Debian standard libjpeg-progs package), available as
     Debian source form from

	http://www.inf.bme.hu/~pts/tif22pnm-latest.tar.gz

Q31. Why don't use libjpeg/libtiff/libpng/zlib or any other library with
     sam2p?

A31. -- library and .h incompatibilities
     -- to avoid forced dependencies
     -- checkergcc wouldn't work

Q32. How do I specify the page size when printing a .ps file (-m and -e
     command line options)?

A32. You cannot. (Use -m to specify the margins.) The page size is
     autodetected by your printer when the page is
     printed. So you can print the same .ps file on different printers, and the
     margins will be all right on all of them.

     If you really have to specify the page size, edit the .ps file and
     insert the `a4 ' or `letter ' command after the last %% line.

Q33. How do I control ZIP compression ratio?

A33. Use

	sam2p -c:zip:1:0 in.img out.png  # uncompressed
	sam2p -c:zip:1:1 in.img out.png  # normal compression
	sam2p -c:zip:1:9 in.img out.png  # maximum compression

Q34. ImageMagick convert creates smaller PNGs. Why?

     You are probably trying to convert a JPEG or other true color photo to
     PNG. Try one of the following compression options:

	sam2p -c:zip:12:7   # 464727 bytes
	sam2p -c:zip:12:8   # 454271 bytes
	sam2p -c:zip:12:9   # 447525 bytes
	sam2p -c:zip:13:9   # 488748 bytes
	sam2p -c:zip:14:9   # 454182 bytes
	sam2p -c:zip:15:9   # 453080 bytes
	convert -quality ?  # 454438 bytes

Q35. Can can sam2p create convert large JPEGs to smaller ones (with loss of
     quality and resolution)?

A35. sam2p cannot resize or scale images. So the pixel width and height of
     the input and output image cannot be changed. If you need that (for
     example you want to create thumbnails), use the famous convert(1)
     utility of ImageMagick. For example:

	convert -scale 444   -quality 50 in.jpg out.jpg  # specify out width
	convert -scale "10%" -quality 50 in.jpg out.jpg  # specify scale ratio

     However, it is possible to specify the quality of the JPEG output of
     sam2p. The
     quality of 0 means ugly output with small file size, and the quality of
     100 means nice output with big file sizes. You can specify intermediate
     integer quality values (50 and 75 are recommended). For example:

	 sam2p -c ijg:10 large_input.jpg small_output.jpg

     sam2p uses the cjpeg(1) and djpeg(1) utilities from libjpeg to write
     and read JPEG files, respectively. If you need more control over your
     JPEG output, then forget sam2p, and please consult the documentation of
     those utilities.

Q36. Can sam2p convert JPEG to GIF?

A36. Yes, it can, but usually not directly. GIF allows maximum 256 different
     colors in an image. A typical RGB JPEG image contains much more colors,
     so it has to be quantized down to 256 colors first. For example, if the
     console output of ``sam2p in.jpg out.gif'' contains ``applyProfile:
     invalid combination, no applicable OutputRule'', then in.jpg must be
     quantized first:

	convert in.jpg out1.gif  # does the quantization automatically
	sam2p out1.gif out2.gif  # compresses the output image further

     From out1.gif and out2.gif keep the one with the smaller file size. It
     is common that convert(1) creates huge GIF files because LZW compression
     is disabled inside it. sam2p should be compiled with LZW compression
     and GIF input/output enabled. To check this, run sam2p, and
     examine its console output. It should contain a line:

	Available Appliers: ... GIF89a+LZW ...

     If GIF89a+LZW doesn't appear in the line, please recompile sam2p with:

	make clean
	./configure --enable-gif --enable-lzw
	make
	cp sam2p /usr/local/bin

     , and try again.

Q37. I need transform gif 15 colors images to BMP 256 colors not
     compressed. sam2p convert it to BMP 16 colors...

A37. Use

	sam2p -c none -s rgb8 in.gif out.bmp

     If you get an error message `Error: applyProfile: invalid combination,
     no applicable OutputRule', it very probably means that your GIF is
     transparent. Remove the transparent color within an image editor first.

Q38. Can sam2p _load_ PDF or EPS files?

A38. Yes, if you have Ghostscript installed, and your input EPS file is not
     too exotic. This has been tested on Linux only. If you experience
     problems loading EPS files, but no problems loading PDF files, please
     run a2ping.pl written by the author of sam2p to make your EPS file more
     compatible.

Q39. Can sam2p load and and EPS file with an arbitrary resoultion?

A39. Yes. For example use

	sam2p -l:gs=-r216 in.eps out.png

     to have resoultion 216 DPI (image scaled 3 times to both directions).
     Without scaling, 72 DPI is the default.

Q40. Can sam2p emit a multi-page PDF or a multi-page PS?

A40. No it can't. Emitting a multi-page document would need a fundamental
     change of the sam2p architecture. (Should the 2nd page compressed with
     a different method? What if the 2nd contains too many colors? Should we
     keep all previous pages in memory?)

     I think another program should be written that is able to concatenate
     EPS/PS or PDF files. I've already written PDF-merger (not released
     yet), and an EPS-merger would be even easier. But I don't have time to
     implement these features soon.

Q41. Can sam2p read a multi-page TIFF?

A41. sam2p reads only the first page.

     The auxilary utility tif22pnm could be patched so it extracts other
     pages, and a new command line option can be added to sam2p that passes
     the required page number to tif22pnm. But I don't have time to
     implement these features soon.

     By the way, multi-page TIFFs can be created with the following command:

	tiffcp -c g4 d1.tiff d2.tiff d3.tiff output.tiff

Q42. Can sam2p convert a multi-page TIFF to a multi-page PDF?

A42. No. There are two main problems: See also Q40 and Q41.

Upsampling
~~~~~~~~~~
Here is a figure about upsampling samples bits 1 -> 2 -> 4 -> 8:

	0 -> 00			0
	1 -> 11			3

	00 -> 0000		0
	01 -> 0101		5
	10 -> 1010		10
	11 -> 1111		15

	0000 -> 0000 0000	0
	0001 -> 0001 0001	17
	0010 -> 0010 0010	34
	...
	1110 -> 1110 1110	238
	1111 -> 1111 1111	255

A 1-bit image has a palette of 2 colors: #00 and #ff.

A 2-bit image has a palette of 4 colors: #00, #55, #aa and #ff.

A 4-bit image has a palette of 16 colors: #00, #11, #22, ... #ff.

An 8-bit image has a palette of 256 colors: #00, #01, #02, ... #ff.

Standards
~~~~~~~~~
-- PSL1 is PostScript LanguageLevel1, as defined by Adobe's PostScript
   Language Reference Manual.
-- PSLC is PSL1 with the CMYK extension (including the `colorimage'
   operator). Supersedes PSL1.
-- PSL2 is PostScript LanguageLevel2, as defined by Adobe's PostScript
   Language Reference Manual. Supersedes PSLC.
-- PSL3 is PostScript LanguageLevel3, as defined by Adobe's PostScript
   Language Reference Manual (PLRM.pdf). Supersedes PSL2.
-- PDF1.0 is PDF version 1.0, as defined by Adobe's PDF Reference.
-- PDF1.1 is PDF version 1.1, as defined by Adobe's PDF Reference.
   Supersedes PDF1.0.
-- PDF1.2 is PDF version 1.2, as defined by Adobe's PDF Reference.
   Supersedes PDF1.1.
-- PDF1.3 is PDF version 1.3, as defined by Adobe's PDF Reference.
   Supersedes PDF1.2.
-- PDF1.4 is PDF version 1.4, as defined by Adobe's PDF Reference
   (PDFRef.pdf). Supersedes PDF1.3.
-- PBM is Portable Bitmap file format, as defined in NetPBM's pbm(5) UNIX
   manual page.
-- PGM is Portable Graymap file format, as defined in NetPBM's pgm(5) UNIX
   manual page.
-- PPM is Portable Pixmap file format, as defined in NetPBM's ppm(5) UNIX
   manual page.
-- PNM is Portable Anymap file format, as defined in NetPBM's pnm(5) UNIX
   manual page. It is the union of PGM, PPM and PPM.
-- PAM is the new, Portable ...map file format, as defined in NetPBM's pam(5)
   UNIX manual page. We don't support it yet.
-- TIFF is ... v6.0.
-- JPEG is baseline JPEG JFIF file format as defined by the Joint Picture
   Expert Group.
-- PNG is Portable Network Graphics file format v1.0, as defined by
   RFC 2083.


Obsolete: Image metadata
~~~~~~~~~~~~~~~~~~~~~~~~
-- Width: uint
-- Height: uint
-- ColorSpace: enum (see later), determines PixBits
-- PixBits: 1..24
-- Origin: coord
-- Comment: string
-- FileFormat:
-- Predictor: enum (see later), same as PSLanguageLevel1 filter
   /Predictor
   1 -- no predictor
   2 -- TIFF predictor 2 (horizontal differencing)
   10 -- PNG predictor, None function
   11 -- PNG predictor, Sub function
   12 -- PNG predictor, Up function
   13 -- PNG predictor, Average function
   14 -- PNG predictor, Paeth function
   15 -- PNG predcitor, individually chosen for each line
-- PredictorColumns: uint
-- PredictorColors: 1..3
-- PredictorBitsPerComponent: 1, 2, 4, 8
-- CompressionEffort: -1..9 (ZIP)
-- CompressionRecordSize: uint (RLE, Fax: K)
-- Compression: enum
   0 -- None
   1 -- LZW (PSL2 PDF1.0 LZWEncode filter EarlyChange=true, UnitLength=8
        LowBitFirst=false)
   2 -- ZIP (PSL3 PDF1.2 FlateEncode filter without options)
   3 -- RLE (PSL2 PDF1.0 RunLengthEncode filter, similar to TIFF PackBits)
   4 -- Fax (PSL2 PDF1.0 CCITTFaxEncode filter, Uncompressed=true!, K=-1,0,1,
        EndOfLine=false, EncodedByteAlign=false, Columns=..., Rows=0,
        EndOfBlock=true, BlackIs1=false, DamagedRowsBeforeError=0)
   5 -- DCT (PSL2 PDF1.0 DCTEncode, options in JPEG stream)
   6 -- IJG (PSL2 PDF1.0 DCTEncode, options in JPEG stream; the IJG libjpeg
        library is used for compression, respecting the quality value 0..100)
   7 -- JAI (PSL2 PDF1.0 DCTEncode, options in JPEG stream; JPEG-as-is: the
        input file must be a JPEG file -- its contents are transferred
        unmodified into the /DCTDecode JPEG stream)
-- TransferEncoding: enum
   0 -- Binary (RawBits, see pbm(5), pgm(5), ppm(5))
   1 -- ASCII (text, chars: 9,10,13,32..126)
   2 -- Hex ((PSL1), PDF1.0, PSL2 ASCIIHexEncode filter)
   3 -- 85 (PSL2 PDF1.0, ASCII85Encode filter)
   4 -- base64, _not_ implemented
   5 -- quoted-printable, _not_ implemented
   6 -- URLencode, _not_ implemented

Compatibility notes
~~~~~~~~~~~~~~~~~~~
by pts@fazekas.hu at Wed Nov 14 12:14:15 CET 2001
Fri Mar 22 11:48:36 CET 2002
Sat Apr 20 19:57:44 CEST 2002
Fri Feb  7 11:15:39 CET 2003

-- Ghostscript 6.50 has problems with /FileFormat/PDFB1.0
   /SampleFormat/JAI|/IJG/DCT
   (Error: /syntaxerror in ID). The problem has been fixed in Ghostscript
   7.04. With the buggy Ghostscript use /FileFormat/PDF1.0 instead.
-- /FileFormat/PDF[B]1.0 /SampleFormat/Mask|/Indexed1 doesn't
   work on Acrobat Reader 5.0 on Linux: a fully opaque, one-color rectangle is
   painted. This works fine on gs 6.50 and xpdf 1.0, so Acrobat Reader is
   assumed to be buggy.
-- The GIMP 1.0 cannot load PlanarConfig Separated TIFF images of type
   GrayA. (But can load PlanarConfig Contiguous GrayA.)
-- xv cannot display gray TIFF images with transparency.
   xv: Sorry, can not handle 2-channel images.
-- (lib)tiff FAX compression an PS /CCITTFaxEncode have black and white the
   opposite way. So `/CCITTFaxEncode <</BlackIs1 true>>' has to be applied
   when creating a TIFF file.
-- libtiff 3.5.4 doesn't read or write an indexed image with transparency:
   Sorry, can not handle contiguous data with PhotometricInterpretation=1,
   and Samples/pixel=2. (Doesn't work with convert or GIMP.)
-- libtiff 3.5.4 doesn't read or write a gray with transparency:
   Sorry, can not handle contiguous data with PhotometricInterpretation=2,
   and Samples/pixel=2. (Doesn't work with convert, works with GIMP.)
-- libtiff doesn't read or write TIFFTAG_SUBFILETYPE/FILETYPE_MASK +
   TIFFTAG_PHOTOMETRIC/PHOTOMETRIC_MASK. One has to use TIFFTAG_EXTRASAMPLES
   instead.
-- libtiff (and the TIFF file format) supports only /Predictor 1 and
   /Predictor 2, with /Compression/LZW and /Compression/ZIP.
-- libtiff supports only bpc=8 and bpc=16 with /Predictor 2
-- libtiff and most TIFF-handling utils have buggy support for TIFF/JPEG.
   See FAQ answer Q4.
-- libtiff supports only files with all components having the same
   BitsPerSample.
-- acroread 4.0 can display all possible /Predictor values with /Indexed1.
-- Ghostscript 5.50 renders (PDF?) images inaccurately: the last bit of
   the 8-bit palette sometimes gets wrong.
-- Netscape Navigator 4.7 displays transparent PNG images with their bKGD
   (or an arbitrary color if bKGD not present) as a solid background. This
   is a bug.
-- pdftops 0.92 has serious problems displaying images if /Predictor != 1.
   The image will be obscured without an error message. Ghostscript 5.50 and
   Acrobat Reader 4 do not have such problems.
-- Ghostscript 5.50 cannot display a PDF with /ColorSpace[/Indexed/DeviceRGB
   ...]. Acrobat Reader 4, Ghostscript 7.04 and pdftops 0.92 can.
-- /Decode is not required in PDF.
-- GIMP 1.0 completely ignores the PNG tRNS chunk! (Thus it won't recognise
   such a transparency in PNG.) Use `pngtopnm -alpha' instead!
-- pngtopnm honors the PNG bKGD chunk only if called as `pngtopnm -mix'
   (and does mixing)
-- convert honors the PNG bKGD chunk (and does mixing)
-- display doesn't honor the PNG bKGD chunk, but has `-bg' command line option
-- xv honors the PNG bKGD chunk (and does mixing)
-- PDF procsets (subsection 8.1 of PDFRef.pdf)
   /PDF
   /Text
   /ImageB Grayscale images or image masks
   /ImageC Color images
   /ImageI Indexed (color-table) images
-- Ghostscript always requires the /Decode entry in image dicts
-- /DCTEncode and /DCTDecode supports only BitsPerComponent==8.
-- Actually PostScript supports 1,2,4,8,12 BitsPerComponent. PDF1.3 supports
   only 1,2,4,8. We support only 1, 2, 4 and 8.
-- PostScript also supports the CMYK color space, not just gray and RGB.
   (And also the HSB, which can be transformed to RGB in an ugly way.)
-- PostScript supports PNG predictors to enhance compression.
-- The PLRM 4.10.6 describes a trick with patterns and imagemask to do
   transparent images. Unfortunately this doesn't work in Ghostscript 5.50
   and xpdf 0.92 (but it works in acroread 4.0), so we don't use it. That's
   why we have only two *-transparent-* entries.
-- ImageMagick EPSI is an EPS with preview (%%BeginPreview .. %%EndPreview)
-- ImageMagick EPSF and EPS are equivalent
-- ImageMagick EPS* is incredibly slooow because of the bad design, even for
   LanguageLevel 2.
-- ImageMagick EPS* cannot display color images without the colorimage
   opertor. (We could do some trickery with multiple calls to imagemask.)
-- tiff2ps cannot display color images without the colorimage
   opertor. (We could do some trickery with multiple calls to imagemask.)
-- Timing: 1495 x 935 RGB, gs -sDEVICE=bmp16m -sOutputFile=/dev/null
   time gs -q -sDEVICE=ppmraw -sOutputFile=t.ppm $IN.eps </dev/null
   ImageMagick 6620 ms user
   tiff2ps-readhexstring 2320 ms user
   currentfile-colorimage 2120 ms user
   readstring-colorimage 2170 ms user
   currentfile-/FlateDecode-colorimage 2670 ms user
-- There is a NullEncode filter, but NullDecode doesn't exist
-- speed conclusions:
   1. Use currentfile as data source (LanguageLevel2) if possible.
   2. /FlateDecode adds a 25% speed penalty. But it compresses quite well,
      so use it!
-- PostScript LanguageLevel2 supports the indexed color space:

	/colormap colors 3 mul string def
	currentfile colormap readhexstring pop pop
	[ /Indexed /DeviceRGB colors 1 sub colormap ] setcolorspace

-- EPS comments: (ImageMagick)

	%%DocumentData: Clean7Bit
	%%LanguageLevel: 1

-- PSL1/PSL2 supports the following color setting operators: (all operands
   are between 0.0 and 1.0)

	<num> setgray  currentgray % PSL1, 0.0=black 1.0=white
	<hue> <saturation> <brightness> sethsbcolor % PSL1
	<red> <green> <blue> setrgbcolor % PSL1
	<cyan> <magenta> <yellow> <black> setcmykcolor % PSL2; not in PSL1

Algorithms, code comments
~~~~~~~~~~~~~~~~~~~~~~~~~
RGB -> Gray
"""""""""""
   gray = 0.299*red + 0.587*green + 0.114*blue (NTSC video std)
   gray = (306.176*red + 601.088*green + 116.736*blue) >> 10
   perl -e 'no integer;for(@ARGV){y/#//d;$N=hex($_);$R=0.299*($N>>16)+0.587*(($N>>8)&255)+0.114*($N&255);print $R/255," ",sprintf("#%02x",int($R+0.5)),"\n"}' '#0f0f0f'
 
Rgb2 -> Gray
""""""""""""
   perl -e 'no integer;for(@ARGV){print"$_
   ";y/#//d;$N=hex($_);$R=0.299*($N>>16)+0.587*(($N>>8)&255)+0.114*($N&255);
   print "",$R/255," ",sprintf("#%02x",int($R+0.5)),"\n"}' 000000 000055 0000aa
   0000ff  005500 005555 0055aa 0055ff  00aa00 00aa55 00aaaa 00aaff  00ff00
   00ff55 00ffaa 00ffff  550000 550055 5500aa 5500ff  555500 555555 5555aa
   5555ff  55aa00 55aa55 55aaaa 55aaff  55ff00 55ff55 55ffaa 55ffff  aa0000
   aa0055 aa00aa aa00ff  aa5500 aa5555 aa55aa aa55ff  aaaa00 aaaa55 aaaaaa
   aaaaff  aaff00 aaff55 aaffaa aaffff  ff0000 ff0055 ff00aa ff00ff  ff5500
   ff5555 ff55aa ff55ff  ffaa00 ffaa55 ffaaaa ffaaff  ffff00 ffff55 ffffaa
   ffffff

000000 0 #00
000055 0.038 #0a
0000aa 0.076 #13
0000ff 0.114 #1d
005500 0.195666666666667 #32
005555 0.233666666666667 #3c
0055aa 0.271666666666667 #45
0055ff 0.309666666666667 #4f
00aa00 0.391333333333333 #64
00aa55 0.429333333333333 #6d
00aaaa 0.467333333333333 #77
00aaff 0.505333333333333 #81
00ff00 0.587 #96
00ff55 0.625 #9f
00ffaa 0.663 #a9
00ffff 0.701 #b3
550000 0.0996666666666667 #19
550055 0.137666666666667 #23
5500aa 0.175666666666667 #2d
5500ff 0.213666666666667 #36
555500 0.295333333333333 #4b
555555 0.333333333333333 #55
5555aa 0.371333333333333 #5f
5555ff 0.409333333333333 #68
55aa00 0.491 #7d
55aa55 0.529 #87
55aaaa 0.567 #91
55aaff 0.605 #9a
55ff00 0.686666666666667 #af
55ff55 0.724666666666667 #b9
55ffaa 0.762666666666667 #c2
55ffff 0.800666666666667 #cc
aa0000 0.199333333333333 #33
aa0055 0.237333333333333 #3d
aa00aa 0.275333333333333 #46
aa00ff 0.313333333333333 #50
aa5500 0.395 #65
aa5555 0.433 #6e
aa55aa 0.471 #78
aa55ff 0.509 #82
aaaa00 0.590666666666667 #97
aaaa55 0.628666666666667 #a0
aaaaaa 0.666666666666667 #aa
aaaaff 0.704666666666667 #b4
aaff00 0.786333333333333 #c9
aaff55 0.824333333333333 #d2
aaffaa 0.862333333333333 #dc
aaffff 0.900333333333333 #e6
ff0000 0.299 #4c
ff0055 0.337 #56
ff00aa 0.375 #60
ff00ff 0.413 #69
ff5500 0.494666666666667 #7e
ff5555 0.532666666666667 #88
ff55aa 0.570666666666667 #92
ff55ff 0.608666666666667 #9b
ffaa00 0.690333333333333 #b0
ffaa55 0.728333333333333 #ba
ffaaaa 0.766333333333333 #c3
ffaaff 0.804333333333333 #cd
ffff00 0.886 #e2
ffff55 0.924 #ec
ffffaa 0.962 #f5
ffffff 1 #ff

Building the palette for Paletted image -> PSLC colorimage
""""""""""""""""""""""""""""""""""""""""""""""""""""""""""
{ Buf3 0 0 currentfile Bufx readstring pop
  % /i 0 def
  % Stack: Buf3 0 0 Bufx
  { % Stack: Buf3 <Buf3-dst-i> <0-1-2> <byte-read>
    exch
    { { % Stack: Buf3 <Buf3-dst-i> <byte-read>
        dup -2 bitshift Pal exch get
	% Stack: Buf3 <Buf3-dst-i> <byte-read> <byte-to-write>
        exch 4 bitshift /Carry exch def
	% Stack: Buf3 <Buf3-dst-i> <byte-to-write>
	3 copy put  pop 1 add  1
	% Stack: Buf3 <Buf3-dst-i+1> 1
      }{dup -4 bitshift Carry add Pal exch get
        exch 15 and 2 bitshift /Carry exch def
	3 copy put  pop 1 add  2
      }{dup -6 bitshift Carry add Pal exch get
	% Stack: Buf3 <Buf3-dst-i> <byte-read> <byte-to-write-first>
	exch 4 copy
	% Stack: Buf3 <Buf3-dst-i> <byte-to-write-first> <byte-read>  Buf3 <Buf3-dst-i> <byte-to-write-first> <byte-read>
        pop put
	% Stack: Buf3 <Buf3-dst-i> <byte-to-write-first> <byte-read>
        63 and Pal exch get  pop
	% Stack: Buf3 <Buf3-dst-i> <byte-to-write-second>
	exch 1 add exch
	3 copy put  pop 1 add  0
	% Stack: Buf3 <Buf3-dst-i+2> 0
      }
    } exch
    % Stack: Buf3 <Buf3-dst-i> <byte-read> {code3} <0-1-2>
    get exec
    % Stack: Buf3 <Buf3-dst-i++> <1-2-0>
  } forall
  % Stack: Buf3 <i>
  pop pop
  Bufr
}

Color space conversion: (unfinished)
""""""""""""""""""""""""""""""""""""
-- Ripped from jccolor.c from libjpeg:

   * YCbCr is defined per CCIR 601-1, except that Cb and Cr are
   * normalized to the range 0..MAXJSAMPLE rather than -0.5 .. 0.5.
   * The conversion equations to be implemented are therefore
   *      Y  =  0.29900 * R + 0.58700 * G + 0.11400 * B
   *      Cb = -0.16874 * R - 0.33126 * G + 0.50000 * B  + CENTERJSAMPLE
   *      Cr =  0.50000 * R - 0.41869 * G - 0.08131 * B  + CENTERJSAMPLE
   * (These numbers are derived from TIFF 6.0 section 21, dated 3-June-92.)



   -- RGB -> YCbCr:
          Y  =  0.29900 * R + 0.58700 * G + 0.11400 * B
          Cb = -0.16874 * R - 0.33126 * G + 0.50000 * B  + CENTERJSAMPLE
          Cr =  0.50000 * R - 0.41869 * G - 0.08131 * B  + CENTERJSAMPLE
   -- CMYK -> YCCK:
          R=1-C, G=1-M, B=1-Y, K=K, RGB -> YCbCr
   -- YCbCr -> RGB: (Cb < CENTERJSAMPLE, Cr < CENTERJSAMPLE)
          R = Y                + 1.40200 * Cr
          G = Y - 0.34414 * Cb - 0.71414 * Cr
          B = Y + 1.77200 * Cb
   -- YCCK -> CMYK:
          YCbCr -> RGB, C=1-R, M=1-G, Y=1-B, K=K.
   -- (YCbCrK == YCCK)
   -- Gray -> RGB: red = gray, green = gray, blue = gray
   -- RGB -> Gray: gray = 0.299*red + 0.587*green + 0.114*blue (NTSC video std)
                 : gray = (306.176*red + 601.088*green + 116.736*blue) >> 10
                 : gray = (19595.264*red + 38469.632*green + 7471.104*blue) >> 16
   -- Obsolete: RGB -> YCbCr: (YCbCr == YUV)
        y:luminance = 0.299*red + 0.587*green + 0.114*blue (0..1)
	cb:chrominance-b = blue - y = - 0.299*red - 0.587*green + 0.886*blue  (-0.886 .. 0.886)
	cr:chrominance-r = red - y = 0.701*red - 0.587*green - 0.114*blue (-0.701 .. 0.701)
      Chrominance components can be represented less accurately since the
      human eye is around twice as sensitive to luminance as to chrominance.
   -- RGB <-> HSB: quite obfuscated. See gshsb.c in Aladdin Ghostscript's
      source.

Word frequency counting in PostScript source
""""""""""""""""""""""""""""""""""""""""""""
perl -e '$C=0; $X=join"",<STDIN>; for (split" ",$X) { print "$_.\n";
$C+=length($_)-1 if /[0-9]/ }; print STDERR "C=$C\n"' 

</tmp/t perl -e '$C=0; $X=join"",<STDIN>; $X=~s@([{}])@ $1 @g; $X=~s@/@ /@g;
for (split" ",$X) { print "$_.\n" if length>=2; $C++ if length($_)>=2 };
print STDERR "C=$C\n"' | sort | uniq

</tmp/t perl -e '$C=0; $X=join"",<STDIN>; $X=~s@([{}])@ $1 @g; $X=~s@/@ /@g;
$Y=""; for (split" ",$X) { $_=" $_ " if length($_)>=2; $Y.=$_ }; $Y=~s@
/@/@g; print $Y' >t

How tiff2ps display gray version of PSLC Rgb8 image on PSL1
"""""""""""""""""""""""""""""""""""""""""""""""""""""""""""
gsave
100 dict begin
1495.000000 935.000000 scale
/bwproc {
    rgbproc
    dup length 3 idiv string 0 3 0
    5 -1 roll {
	add 2 1 roll 1 sub dup 0 eq {
	    pop 3 idiv
	    3 -1 roll
	    dup 4 -1 roll
	    dup 3 1 roll
	    5 -1 roll put
	    1 add 3 0
	} { 2 1 roll } ifelse
    } forall
    pop pop pop
} def
/colorimage where {pop} {
    /colorimage {pop pop /rgbproc exch def {bwproc} image} bind def
} ifelse
%ImageData: 1495 935 8 3 0 1 2 "false 3 colorimage"
/line 4485 string def
1495 935 8
[1495 0 0 -935 0 935]
{currentfile line readhexstring pop} bind
false 3 colorimage

pts_defl.c which function calls which function
""""""""""""""""""""""""""""""""""""""""""""""
send_bits
  rengeteg
flush_outbuf
  send_bits
  flush_block 2*
build_tree
  flush_block 3*
flush_block
  deflate2 5*
ct_tally
  deflate2 sok*
longest_match
  deflate2 2*
fill_window
  deflate2 4*

gen_codes
  2, reentrant
deflate2
  1..
pts_deflate_init
  1

Answers from Taco
"""""""""""""""""
Are the problems I'm dealing about already solved?

   -- Is there a pixel image converter that can create small and compatible
      EPS documents with all the RunLength, CCITTFax, LZW and Flate
      compression types?

   -- Is there a PDF version of psfrag.sty? Is there a utility for splitting
      and assebling PDF files? Is there a utility for changing text labels
      inside a PDF file?

   -- Is there a utility for creating a (semi-)transparent EPS or PDF files
      from pixel images?

Compilation problems
""""""""""""""""""""
M�rten Svantesson wrote:

> Another problem was that the configure script didn't realise that it
> couldn't use the forte CC-compiler. This lead to several errors. For
> example it don't accept the option -ansi, so several tests fails of
> this reason.

I don't have access to the a forte CC-compiler. The configure script has
been designed for use with GCC and G++.

History
~~~~~~~
0.29 Apr 4 2002
     sam2p_article.tex preliminary version
0.30 Sat Apr  6 09:19:23 CEST 2002
0.31 -- Fri Apr 12 23:54:57 CEST 2002
     not working TCL/TK GUI, sam2p_article corrected
0.32 Sat Apr 13 12:55:09 CEST 2002 --
     PCX input, class Encoder, class Decoder
0.33 JPEG, TIFF, PNG input. JPEG output.
0.34 PDF output, PNG output.
0.35 Fri Apr 26 08:28:41 CEST 2002
     transparency, gui, sam2p_talk, EuroBachoTeX 2002
0.36 Sat May 26 14:06:09 CEST 2002
     Blanca bugfixes
0.37 Sat Jun  1 14:06:43 CEST 2002
     ccdep.pl, --enable-debug, XPM output, opaque TIFF output,
     more PS output, BMP output, transparent PNM output, one-liner mode,
     XWD output, PS output with margins, autoconf consts,
     built-in CCITTFaxEncode
0.38 Tue Sep  3 20:16:59 CEST 2002
     Many portability bugfixes. Now more versions of GNU C++ compilers and
     more achitectures are tested and supported.
     gcc-3.1 or gcc-3.2 compliance (AC_C_CONST working const; with autoconf
     2.53)
0.39 -j:warn, allow /OutputFile(-) (stdout) in one-liner mode,
     one-liner mode with src == dst filename `--', EPS margins, PDF margins,
     -m, scaling PS output command line, PSL1 Gray ZIP&LZW in all
     TransferEncodings

0.40 Wed Dec 11 19:24:31 CET 2002
     builds out-of-the box (./configure ...; make -> sam2p.exe) on Win32 with MinGW32 + Perl

0.41 GIF and XPM load bugfixes
     synchronized to GIMP tif22pnm
     transparent(+?) TIFF output, really
     better tif22pnm support

0.42 Wed Feb  5 18:40:12 CET 2003
     vcsam2p.exe (graphic .exe with Visual Studio 6.0), merged
     added Below feature -m:lower:5
     png22pnm support
     Debian Slink pdfTeX integration

0.43 Fri Feb  7 11:32:10 CET 2003
     PDF generation bugfixes
     PDF /MediaBox is output into a separate line near the beginning of file (graphicP compatibility)

Further changes moved to debian/changelog.

Missing feature list
~~~~~~~~~~~~~~~~~~~~
!! make mkdist-noautoconf creates .tar.gz w/o leading directory part
!! cannot read t.pnm "P4 1 1 1" doesn't end with \S
!! ./sam2p ../pts.ppm t.gif -> sam2p: out_gif.cpp:260: void out_gif_write(GenBuffer::Writable&, Image::Indexed*): Assertion `bits_per_pixel<=8' failed.
!! why so large PNG? sam2p examples/chessboard_ascii.pnm t.png
!! easy seek back to 0 in in_*.cpp
!! rxvt_bug.png -> useful Win98 .bmp
??
     (expected) !!
     --invert option, for tif22pnm users
     complete turbo tutorial,
     MIFF input, _output_
     full BMP output compressed,
     faster GIF input??, make install, .deb (debhelper?), .rpm,
     normal `image' instead of `colorimage'
     CUPS raster input and output,
     PS and PDF input,
     /Background option (-bg) to change transparent pixels -> bg color
     LZW and ZIP compressed for indexed etc.
/* Imp: command line option for /Transparent */
/* Imp: avoid 3 warnings in: ./sam2p examples/pts2.pbm t.ppm */
/* OK : margins for all PS, EPS and PDF (Sat Sep  7 16:17:58 CEST 2002 */
/* Imp: howto in pts_defl.h */
/* Imp: check for fixed bugs new zlib */
/* OK : built-in CCITTFaxEncode */
/* Imp: one-liner eliminate automatically invalid BMP SampleFormats with compression */
/* OK : more transferencodings for l1fa85g.tte */
/* Imp: eliminate long line (>120) from l1fa85g.tte */
/* Imp: one-liner -s gray|indexed|rgb|transparent */
/* OK : read bad.txt, and move it to section {Fixed bugs} */
/* OK : verify /Compression/Fax /K 1|2 with gs 5.50|7.04 */
/* Imp: in-memory dump when scanf_dict Error */
/* Imp: optimal, non-mem-hungry RLE compression filter (hardly possible...) */
/* Imp: possibility to write RGB (non-YCbCr) JPEG with /Compression/IJG etc. */
/* Imp: ability to load .gz and .zip compressed input files. Example:
 * .pbm.gz
 */
/* Imp: no return DONT_KNOW from *_work() */
/* Imp: Meta output */
/* Imp: PSL1 auto fallback to grayscale `image' in absence of `colorimage' */
/* Imp: full support for ZIP compression in PSL1 (similar to RLE) */
/* OK:  TIFF output */
/* Imp: implement as a library (throw/catch exceptions, memory management, reentrance) */
/* Imp: check for proper inner dict usage in l1fa85g.tte */
/* Imp: add /Transparent2, /Transparent4, /Transparent8 for PDF */
/* Imp: add all #warning REQUIRES: */
/* OK : ccdep.pl, --enable-debug (Sat Jun  1 16:27:58 CEST 2002) */
/* Imp: show binary/ASCII in PDF and PS header comments */
/* Imp: `available loaders: (TIFF)' if tifftopnm not found (run-time) */
/* Imp: make LZW-unsupported not an Error, but a Warning */
/* Imp: create output profile */
/* OK:  a real string hashing lib */
/* Imp: GIF reader is too slow (maybe LZW reading? ?). Change to xvgif.c?? */
/* OK : XPM reader (>=256 colors) is too slow */
/* OK : specify transparent color in the Job file */
/* Imp: run-time detect the absence of external progs */
/* Imp: make install */
/* Imp: make .deb, .rpm */
/* No : NDEBUG by default */
/* Imp: substitute char (1) for bool (8) on Digital UNIX */
/* OK : real replacement of vsnprintf() (Digital UNIX, Solaris) -- Tue Jun 11 15:20:22 CEST 2002 */
/* OK : close both filters in l23.tte, but _never_ close currentfile */
/* Imp: run-time choice from FlateEncode filter implemenetations etc. */
/* Imp: convert bilevel indexed image to grayscale more quickly */
/* OK : add quick and effective command line interface (one-liner mode) */
/* No : read external .tte files (runtime) instead of built-in .tth */
/* Imp: possibility to store several profiles in the same .jib file */
/* Imp: real, PDF-style dates into Created, Produced */
/* OK : real hashing in MiniPS::Dict */
/* Imp: enforce magic numbers for .job and .jib files */
/* OK : /TransferEncoding: * -> ascii if ... */
/* Imp: (early check, special handler) make everything work when wd==0 || ht==0; especially encoding filters */
/* Imp: better Job option for specifying and removing transparent color (better than /Transparent) */
/* Imp: log date(now) */
/* OK : DCT options in Output Rule */
/* OK : implement JAI */
/* Imp: support LargeBBox */
/* Imp: support Comment */
/* Imp: support PDF metadata: Title Subject Author Creator Producer */
/* Imp: support PDF dates: Created Produced */
/* Imp: support Comment */
/* Imp: add PAM file format support from NetPBM */
/* OK : verify ADSC EPSF-3.0 compatibility */
/* Imp: pre-transformation: making grayscale */
/* Imp: pre-transformation: down-bit-sampling */
/* Imp: pre-transformation: down-palette-sampling (``generate optimal palette'') */

==6750== 
==6750== ERROR SUMMARY: 0 errors from 0 contexts (suppressed: 0 from 0)
==6750== malloc/free: in use at exit: 3741 bytes in 102 blocks.
==6750== malloc/free: 2296 allocs, 2194 frees, 170451 bytes allocated.
==6750== For counts of detected errors, rerun with: -v
==6750== searching for pointers to 102 not-freed blocks.
==6750== checked 4227304 bytes.
==6750== 
==6750== 12 bytes in 1 blocks are definitely lost in loss record 3 of 15
==6750==    at 0x400254FF: __builtin_new (vg_replace_malloc.c:172)
==6750==    by 0x806ABC3: MiniPS::Parser::parse1(int, int) (minips.cpp:789)
==6750==    by 0x806B378: MiniPS::Parser::parse1(int, int) (minips.cpp:856)
==6750==    by 0x804DBD5: run_sam2p_engine(Files::FILEW &, Files::FILEW &, char const *const *, bool) (sam2p_main.cpp:923)
==6750== 
==6750== 
==6750== 808 bytes in 81 blocks are possibly lost in loss record 14 of 15
==6750==    at 0x400255E9: __builtin_vec_new (vg_replace_malloc.c:197)
==6750==    by 0x8060C94: Mapping::DoubleHash::set(char const *, unsigned int, char const *) (mapping.cpp:43)
==6750==    by 0x806A18B: MiniPS::Dict::put(char const *, unsigned int, int) (minips.cpp:461)
==6750==    by 0x8069FF1: MiniPS::Dict::put(char const *, int) (minips.cpp:431)
==6750== 
==6750== LEAK SUMMARY:
==6750==    definitely lost: 12 bytes in 1 blocks.
==6750==    possibly lost:   808 bytes in 81 blocks.
==6750==    still reachable: 2921 bytes in 20 blocks.
==6750==         suppressed: 0 bytes in 0 blocks.


Fixed bugs
~~~~~~~~~~
(this section was formerly bad.txt)

./sam2p -j examples/ptsbanner2.jpg test.pdf
  discover at Sun Sep 22 14:38:39 CEST 2002
  BUGFIX at Sun Sep 22 15:03:33 CEST 2002

./sam2p -1 -s:rgb1 -c:none examples/pts2.pbm test.eps
  discover at Sun Sep 22 14:29:59 CEST 2002
  BUGFIX (add -s:...:stopq) at Sun Sep 22 16:20:41 CEST 2002
  (behaviour correct, specifying -s:rgb1:stop gives more error messages)

./sam2p -c:fax:2 examples/shot.gif y.pdf
  BUGFIX /EndOfLine true is forced
  old gs: /ioerror in --%image_file_continue--
  old acroread: `read less image data' (PDFB1.0: `expected EI')
./sam2p -c:fax:1 examples/shot.gif y.pdf
  BUGFIX /EndOfLine true is forced
  gs: OK
  old acroread: `read less image data' (PDFB1.0: `problem 115')
./sam2p -c:fax:0 examples/shot.gif y.pdf
  gs, acroread: OK
./sam2p -c:fax:-1 examples/shot.gif y.pdf
  gs, acroread: OK

./sam2p -c:fax examples/pts2.pbm y.tiff
  BUGFIX black-white inversion
./sam2p -c:fax examples/shot.gif y.tiff
  BUGFIX black-white inversion

./sam2p -j -c jpeg examples/pts2.pbm y.tiff
  seems to be OK now
./sam2p -j -c jpeg examples/pts2.pbm y.tiff
  seems to be OK now

./sam2p -j -c fax examples/ptsbanner.gif y.eps
  BUGFIX guard against applying /Compression/Fax to non-1-bit data
./sam2p -j -c fax -s:Indexed8 examples/ptsbanner.gif y.eps
  OK, no guard
./sam2p -j -c fax -c none examples/ptsbanner.gif y.eps
  OK, no guard
./sam2p -j -c fax examples/ptsbanner.gif y.tiff
  BUGFIX forbidden applying /Compression/Fax to non-1-bit data

./sam2p -c fax csanyi_torok_old.png csanyi_torok.eps
./sam2p -c fax examples/ptsbanner.gif csanyi_torok.eps
  error: Impossible combination
  OK, the user should specify `-s Indexed8' for unexpected use of  `-c fax'

__END__
